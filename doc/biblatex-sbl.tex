\documentclass{ltxdockit}[2011/03/25]

\usepackage{xcolor}
\usepackage{btxdockit}
\usepackage{fontspec}
\usepackage{xparse}
\usepackage{framed}

\usepackage[style=sbl,backend=biber]{biblatex}
\addbibresource{biblatex-sbl.bib}

\hypersetup{colorlinks,citecolor=spot}

\setmonofont[Scale=MatchLowercase]{DejaVu Sans Mono}
\setromanfont[Ligatures=TeX]{Linux Libertine O}
\setsansfont[Ligatures=TeX]{Linux Biolinum O}

\newcommand*{\biblatexsbl}{\sty{biblatex-sbl}\xspace}
\newcommand*{\biblatexsblhome}{https://github.com/dcpurton/biblatex-sbl/}
\newcommand*{\biblatex}{\sty{biblatex}\xspace}

\ExplSyntaxOn
\NewDocumentCommand \samplemacro { m }
  {
    \texttt{#1}\par
  }
\NewDocumentCommand \sblrefsamplecite { s m m m o o m }
  {
    \IfNoValueTF { #5 }
      {
        \IfNoValueT { #6 }
          {
            \IfBooleanF { #1 }
              {
                \samplemacro{\textbackslash #2\{#7\}}
              }
            \hspace*{\bibindent}#4\csuse{#3}{#7}
          }
      }
      {
        \IfNoValueTF { #6 }
          {
            \IfBooleanF { #1 }
              {
                \samplemacro{\textbackslash #2[#5]\{#7\}}
              }
            \hspace*{\bibindent}#4\csuse{#3}[#5]{#7}
          }
          {
            \IfBooleanF { #1 }
              {
                \samplemacro{\textbackslash #2[#5][#6]\{#7\}}
              }
            \hspace*{\bibindent}#4\csuse{#3}[#5][#6]{#7}
          }
      }
  }
\NewDocumentCommand \samplecite { s m o o m }
  {
    \rmfamily
    \IfBooleanTF { #1 }
      {
        \sblrefsamplecite*{autocite}{cite}{#2.~}[#3][#4]{#5}.\par
      }
      {
        \sblrefsamplecite{autocite}{cite}{#2.~}[#3][#4]{#5}.\par
      }
  }
\NewDocumentCommand \sampleparencite { s o o m }
  {
    \rmfamily
    \IfBooleanTF { #1 }
      {
        \sblrefsamplecite*{parencite}{parencite}{}[#2][#3]{#4}\par
      }
      {
        \sblrefsamplecite{parencite}{parencite}{}[#2][#3]{#4}\par
      }
  }
\NewDocumentCommand \samplebib { s m }
  {
    \IfBooleanF { #1 }
      {
        \samplemacro{\textbackslash printbibliography}
      }
    \hangindent\bibindent\bibentrycite{#2}.\par
  }
\NewDocumentCommand \samplebiblist { s m }
  {
    \IfBooleanF { #1 }
      {
        \samplemacro{\textbackslash printbiblist\{abbreviations\}}
      }
    \biblistcite{#2}
  }
\ExplSyntaxOff

\lstset{%
  basicstyle=\displayverbfont\normalsize
}

\makeatletter
\def\ltd@printarg@iv(#1|#2){[\prm{#1}|\prm{#2}]\ltd@parseargs}
\def\ltd@printarg@v<#1|>{[\prm{#1}|]\ltd@parseargs}

\titlepage{%
  title={\biblatexsbl},
  subtitle={SBL Style Using \biblatex},
  url={\biblatexsblhome},
  author={David Purton},
  email={dcpurton@marshwiggle.net},
  revision={\sbl@abx@version},
  date={\today}}
\makeatother

\hypersetup{%
  pdftitle={biblatex-sbl},
  pdfsubject={SBL Style Using biblatex},
  pdfauthor={David Purton},
  pdfkeywords={sbl, biblatex, bibliography, citation}}

\definecolor{spot}{rgb}{0.25,0.25,0.65}
\colorlet{shadecolor}{black!15}

\begin{document}

\printtitlepage

\tableofcontents

\section{Introduction}

\biblatexsbl provides support to \biblatex and LaTeX for citations,
bibliography, and a list of abbreviations in the style recommended by the
Society of Biblical Literature (\cite{SBL}). The style conforms to the second
edition of the \cite{SBLHS}.

The style supports all examples given in the handbook (see
\sty{biblatex-sbl-test.pdf}). Short form citations and a list of abbreviations
containing series, journals, and shorthands are handled automatically.
\emph{Ibidem} is supported, but not enabled by default. Only note style
citations, not Author-Date citations are supported. Primary sources can be
cited in parentheses. \biblatexsbl is compatible with \biblatex's support for
\sty{hyperref}.

For anything not covered in this manual, please see the \biblatex
documentation. Bugs and feature requests can be submitted at
\url{\biblatexsblhome}.

\section{Requirements}

\biblatexsbl requires at least version 3.0 of \biblatex and the \sty{xparse}
package. \sty{biber} must be used. \sty{bibtex} is not supported. For
localization \sty{babel} and \sty{csquotes} are recommended.

\section{Usage}

The following minimal example will set up \biblatexsbl to conform to the
defaults of the \cite{SBLHS}.

\begin{quote}
\begin{lstlisting}[style=latex]{}
\documentclass{article}
\usepackage[style=sbl,backend=biber]{biblatex}
\addbibresource{<bibfile.bib>}
\begin{document}
\printbiblist{abbreviations}
\printbibliography
\end{document}
\end{lstlisting}
\end{quote}

\subsection{Localization}

By default \biblatexsbl uses American style punctuation and quotation marks.
You can choose a different style by including the \sty{babel} and
\sty{csquotes} packages in your document preamble.
e.g.,

\begin{quote}
\begin{lstlisting}[style=latex]{}
\usepackage[german]{babel}
\usepackage{csquotes}
\usepackage[style=sbl,backend=biber]{biblatex}
\end{lstlisting}
\end{quote}

Currently \opt{english} (including variants such as \opt{british},
\opt{australian}, etc.), \opt{spanish}, and \opt{german} are supported.

\subsection{Commands}

The standard commands for \biblatexsbl generally follow those defined by
\biblatex. Included below are the most typical commands. For more commands and
options, reference the \biblatex manual.

\begin{ltxsyntax}

\cmditem{autocite}{key}
\cmditem{autocite}[postnote]{key}
\cmditem{autocite}[prenote][]{key}
\cmditem{autocite}[prenote][postnote]{key}
\cmditem{autocite}(altpostnote|postnote){key}
\cmditem{autocite}<altpostnote|>{key}
\cmditem{autocite}[prenote](altpostnote|postnote){key}
\cmditem{autocite}[prenote]<altpostnote|>{key}

\cmd{autocite} inserts a citation as a footnote. It works as in the standard
\biblatex styles, except that that \bibfield{postnote} argument can be divided
into two using the pipe (\sty{|}) character. This creates an
\bibfield{altpostnote} field which is used in some of the examples from §6.4
of the \cite{SBLHS}. e.g.,

\begin{snugshade}
  \samplecite{1}[See][1.3|8:223]{clementinehomilies}
\end{snugshade}

\cmditem{parencite}{key}
\cmditem{parencite}[postnote]{key}
\cmditem{parencite}[prenote][]{key}
\cmditem{parencite}[prenote][postnote]{key}
\cmditem{parencite}(altpostnote|postnote){key}
\cmditem{parencite}<altpostnote|>{key}
\cmditem{parencite}[prenote](altpostnote|postnote){key}
\cmditem{parencite}[prenote]<altpostnote|>{key}

\cmd{parencite} works in the same way as \cmd{autocite} except that the
citation is placed inside parentheses instead of in a footnote. This is most
useful for citing primary sources. e.g.,

\begin{snugshade}
  \sampleparencite[2.233-235]{josephus:ant}
\end{snugshade}

\cmditem{seriescite}{key}
\cmditem{journalcite}{key}
\cmditem{shorthandcite}{key}

\cmd{seriescite}, \cmd{journalcite}, and \cmd{shorthandcite} inserts the
respective abbreviation into the text and also adds it to the list of
abbreviations. The abbreviation is hyperlinked to the list of abbreviations if
the \sty{hyperref} package is loaded.

These commands ignore the \bibfield{prenote} and \bibfield{postnote} fields,
so can safely be used anywhere within a database entry.

\cmditem{printbiblist}

This command prints a bibliography list. In \biblatexsbl all abbreviations
(series, journals, and shorthands) can be printed using the following command:

\begin{quote}
  \verb+\printbiblist[...]{abbreviations}+
\end{quote}

See the \biblatex manual for an explanation of available optional arguments.

\cmditem{printbiblioraphy}

Inserts the bibliography. See the \biblatex manual for an explanation of
available optional arguments.

\end{ltxsyntax}

\subsection{Package Options}

\biblatexsbl defaults to the recommendations of the \cite{SBL}, but it also
supports many of the standard options from \biblatex as well as a few custom
ones outlined below.

\begin{optionlist}

\optitem[false]{ibidtracker}{\opt{true}, \opt{false}}

This option controls the \emph{ibidem} tracker. The possible choices are:

\begin{valuelist}
\item[true] Enable the tracker: \emph{ibid.}\ will be used.
\item[false] Disable the tracker: \emph{ibid.}\ will not be used.
\end{valuelist}

\optitem[sbl]{citepages}{\opt{sbl}, \opt{permit}, \opt{omit}, \opt{separate}}

Use this option to fine-tune the formatting of the \bibfield{pages} field
the first time an entry is cited.

\begin{valuelist}
\item[sbl] The \bibfield{postnote} field is not printed for first citations.
  e.g.,

  \begin{snugshade}
    \samplecite{1}[159]{leyerle:1993}
    \samplecite{2}[159]{leyerle:1993}
    \samplecite{3}{leyerle:1993}
  \end{snugshade}

  If \bibfield{postnote} is not a page range, then it is printed in
  parentheses after \bibfield{pages}. e.g.,

  \citereset

  \begin{snugshade}
    \samplecite{1}[a note]{leyerle:1993}
    \samplecite{2}[a note]{leyerle:1993}
  \end{snugshade}

\item[permit] The \bibfield{postnote} is printed in parentheses after the
  \bibfield{pages} field. e.g.,

  \citereset
  \makeatletter\cbx@opt@citepages@permit\makeatother

  \begin{snugshade}
    \samplecite{1}[159]{leyerle:1993}
    \samplecite{2}[159]{leyerle:1993}
  \end{snugshade}

  If \bibfield{postnote} is not a page range, then \bibfield{pages} is printed
  for subsequent citations, and the \bibfield{postnote} is printed in
  parentheses. e.g.,

  \begin{snugshade}
    \samplecite{3}[a note]{leyerle:1993}
  \end{snugshade}

\item[omit] The \bibfield{pages} field is not printed unless
  \bibfield{postnote} is empty or not a page range. e.g.,

  \citereset
  \makeatletter\cbx@opt@citepages@omit\makeatother

  \begin{snugshade}
    \samplecite{1}[159]{leyerle:1993}
    \samplecite{2}[159]{leyerle:1993}
    \samplecite{3}{leyerle:1993}
    \samplecite{4}[a note]{leyerle:1993}
  \end{snugshade}

\item[separate] The \bibfield{postnote} is printed in parentheses after the
  \bibfield{pages} field preceeded by the bibliography string \sty{thiscite}.
  e.g.,

  \citereset
  \makeatletter\cbx@opt@citepages@separate\makeatother

  \begin{snugshade}
    \samplecite{1}[159]{leyerle:1993}
    \samplecite{2}[159]{leyerle:1993}
  \end{snugshade}

  If \bibfield{postnote} is not a page range, then \sty{firstcite} is not
  used. e.g.,

  \begin{snugshade}
    \samplecite{3}[a note]{leyerle:1993}
  \end{snugshade}

  \makeatletter\cbx@opt@citepages@sbl\makeatother

\end{valuelist}

\boolitem[false]{fullbibrefs}

The \emph{Student Supplement for the} \cite{SBLHS} permits two styles for the
bibliography entry for Bible dictionaries and encyclopaedias, and multivolume
commentaries for the entire Bible by multiple
authors.\autocite[4–5]{SBLHS:studentsupp}

This option applies to \bibfield{@inreference} and \bibfield{@incommentary}
entry types.

\begin{valuelist}
\item[true] The bibliography entry is printed in long form. e.g.,

  \begin{snugshade}
    \toggletrue{fullbibrefs}
    \nocite{IDB}
    \samplebib*{stendahl:1962}
    \togglefalse{fullbibrefs}
  \end{snugshade}

\item[false] The bibliography entry is printed in a short form. e.g.,

  \begin{snugshade}
    \samplebib*{stendahl:1962}
  \end{snugshade}
\end{valuelist}

\boolitem[true]{sblfootnotes}

This option controls the style of footnotes.

\begin{valuelist}
\item[false] Footnotes are printed with a superscript and hanging indent (or
  whatever other default has been set up by your style):

  \begin{snugshade}
    \hangindent\bibindent\textsuperscript{1}\cite{talbert:1992}.
  \end{snugshade}

\item[true] Footnotes are printed with a with a normal number followed by a
  period and the first line indented:

  \begin{snugshade}
    \citereset
    \samplecite*{1}{talbert:1992}
    \citereset
  \end{snugshade}
\end{valuelist}

\end{optionlist}

\section{Database Guide}

\printbiblist[heading=biblistintoc]{abbreviations}

\printbibliography[heading=bibintoc]

\end{document}
