\documentclass{ltxdockit}[2011/03/25] 
\usepackage{microtype}
\usepackage{xcolor}
\usepackage{makeidx}
\makeindex
\usepackage[totoc]{idxlayout}
\usepackage{btxdockit}
\usepackage{fontspec}
\usepackage{realscripts}
\usepackage{xparse}
\usepackage{framed}

\usepackage[style=sbl,indexing=cite,backend=biber]{biblatex}
\addbibresource{biblatex-sbl.bib}

\renewcommand{\indexname}{Index of Authors}

\hypersetup{colorlinks,citecolor=spot}

\hyphenation{Prei-sen-daz}

\setmonofont{DejaVu Sans Mono}[Scale=MatchLowercase]
\setromanfont{Linux Libertine O}
\setsansfont{Linux Biolinum O}[
  BoldItalicFont={* Bold},
  BoldItalicFeatures={FakeSlant=0.2}
]

\newcommand*{\biblatexsbl}{\sty{biblatex-sbl}\xspace}
\newcommand*{\biblatexsblhome}{https://github.com/dcpurton/biblatex-sbl/}
\newcommand*{\biblatex}{\sty{biblatex}\xspace}

\ExplSyntaxOn
\NewDocumentCommand \samplemacro { m }
  {
    \texttt{#1}\par
  }
\NewDocumentCommand \sblrefsamplecite { s m m m o o m }
  {
    \IfNoValueTF { #5 }
      {
        \IfNoValueT { #6 }
          {
            \IfBooleanF { #1 }
              {
                \samplemacro{\textbackslash #2\{#7\}}
              }
            \hspace*{\bibindent}#4\csuse{#3}{#7}
          }
      }
      {
        \IfNoValueTF { #6 }
          {
            \IfBooleanF { #1 }
              {
                \samplemacro{\textbackslash #2[#5]\{#7\}}
              }
            \hspace*{\bibindent}#4\csuse{#3}[#5]{#7}
          }
          {
            \IfBooleanF { #1 }
              {
                \samplemacro{\textbackslash #2[#5][#6]\{#7\}}
              }
            \hspace*{\bibindent}#4\csuse{#3}[#5][#6]{#7}
          }
      }
  }
\NewDocumentCommand \samplecite { s m o o m }
  {
    \rmfamily
    \IfBooleanTF { #1 }
      {
        \sblrefsamplecite*{autocite}{cite}{#2.~}[#3][#4]{#5}.\par
      }
      {
        \sblrefsamplecite{autocite}{cite}{#2.~}[#3][#4]{#5}.\par
      }
  }
\NewDocumentCommand \sampleparencite { s o o m }
  {
    \rmfamily
    \IfBooleanTF { #1 }
      {
        \sblrefsamplecite*{parencite}{parencite}{}[#2][#3]{#4}\par
      }
      {
        \sblrefsamplecite{parencite}{parencite}{}[#2][#3]{#4}\par
      }
  }
\NewDocumentCommand \samplebib { s m }
  {
    \IfBooleanF { #1 }
      {
        \samplemacro{\textbackslash printbibliography}
      }
    \hangindent\bibindent\bibentrycite{#2}.\par
  }
\NewDocumentCommand \samplebiblist { s m }
  {
    \IfBooleanF { #1 }
      {
        \samplemacro{\textbackslash printbiblist\{abbreviations\}}
      }
    \biblistcite{#2}
  }
\ExplSyntaxOff

\lstset{%
  basicstyle=\displayverbfont\normalsize,
  keywordstyle=\bfseries
}

\makeatletter
\def\ltd@printarg@iv(#1|#2){[\prm{#1}|\prm{#2}]\ltd@parseargs}
\def\ltd@printarg@v<#1|>{[\prm{#1}|]\ltd@parseargs}

\titlepage{%
  title={\biblatexsbl},
  subtitle={SBL Style Using \biblatex},
  url={\biblatexsblhome},
  author={David Purton},
  email={dcpurton@marshwiggle.net},
  revision={\sbl@abx@version},
  date={\today}}
\makeatother

\hypersetup{%
  pdftitle={biblatex-sbl},
  pdfsubject={SBL Style Using biblatex},
  pdfauthor={David Purton},
  pdfkeywords={sbl, biblatex, bibliography, citation}}

\definecolor{spot}{rgb}{0.25,0.25,0.65}
\colorlet{shadecolor}{black!15}

\begin{document}

\printtitlepage

\tableofcontents

\section{Introduction}

\biblatexsbl provides support to \biblatex and LaTeX for citations,
bibliography, and a list of abbreviations in the style recommended by the
Society of Biblical Literature (\citeshorthand{SBL}). The style conforms to
the second edition of the \cite{SBLHS}.

The style supports all examples given in the handbook (see
\sty{biblatex-sbl-test.pdf}). Shorthand citations and a list of abbreviations
containing journals, series, and shorthands are handled automatically.
Repeated authors in the bibliography are replaced by a horizontal line.
\emph{Ibidem} is supported, but not enabled by default, as is indexing of
names. Only note style citations, not Author-Date citations are supported.
Primary sources can be cited in parentheses. \biblatexsbl is compatible with
\biblatex's support for \sty{hyperref}.

For anything not covered in this manual, please see the \biblatex
documentation. Bugs and feature requests can be submitted at
\url{\biblatexsblhome}.

\section{Requirements}

\biblatexsbl requires at least version 3.0 of \biblatex and the \sty{xparse}
package. \sty{biber} must be used. \sty{bibtex} is not supported. For
localization \sty{babel} and \sty{csquotes} are recommended.

\section{Usage}

The following minimal example will set up \biblatexsbl to conform to the
defaults of the \cite{SBLHS}.

\begin{quote}
\begin{lstlisting}[style=latex]{}
\documentclass{article}
\usepackage[style=sbl,backend=biber]{biblatex}
\addbibresource{<bibfile.bib>}
\begin{document}
\printbiblist{abbreviations}
\printbibliography
\end{document}
\end{lstlisting}
\end{quote}

\subsection{Localization}

By default \biblatexsbl uses American style punctuation and quotation marks.
You can choose a different style by including the \sty{babel} and
\sty{csquotes} packages in your document preamble.
e.g.,

\begin{quote}
\begin{lstlisting}[style=latex]{}
\usepackage[german]{babel}
\usepackage{csquotes}
\usepackage[style=sbl,backend=biber]{biblatex}
\end{lstlisting}
\end{quote}

Currently \opt{english} (including variants such as \opt{british},
\opt{australian}, etc.), \opt{spanish}, and \opt{german} are supported.

\subsection{Commands}

The standard commands for \biblatexsbl generally follow those defined by
\biblatex. Included below are the most typical commands. For more commands and
options, reference the \biblatex manual.

\begin{ltxsyntax}

\cmditem{autocite}[prenote](altpostnote|postnote){key}

\cmd{autocite} inserts a citation as a footnote. If used in a footnote, the
citation is placed in parentheses. It works as in the standard \biblatex
styles, except that that \bibfield{postnote} argument can be divided into two
using the pipe (\sty{|}) character. This creates an \bibfield{altpostnote}
field which is used in some of the examples from §6.4 of the \cite{SBLHS}.
e.g.,

\begin{snugshade}
  \samplecite{1}[See][1.3|8:223]{clementinehomilies}
\end{snugshade}

To use only \bibfield{altpostnote} you must still include the pipe character.
e.g.,

\begin{snugshade}
  \samplecite{1}[III. 1-164|]{PGM:betz}
\end{snugshade}

\cmditem{cite}[prenote](altpostnote|postnote){key}

\cmd{cite} works in the same way as \cmd{autocite} except that the citation is
placed directly into the text instead of in a footnote.

\cmditem{parencite}[prenote](altpostnote|postnote){key}

\cmd{parencite} works in the same way as \cmd{autocite} except that the
citation is placed inside parentheses instead of in a footnote. This is most
useful for citing primary sources. e.g.,

\begin{snugshade}
  \sampleparencite[2.233-235]{josephus:ant}
\end{snugshade}

\cmditem{journalcite}{key}
\cmditem{seriescite}{key}
\cmditem{shorthandcite}{key}

\cmd{journalcite}, \cmd{seriescite},and \cmd{shorthandcite} inserts the
respective abbreviation into the text and also adds it to the list of
abbreviations. The abbreviation is hyperlinked to the list of abbreviations if
the \sty{hyperref} package is loaded.

These commands ignore the \bibfield{prenote} and \bibfield{postnote} fields,
so can safely be used anywhere within a database entry.

\cmditem{printbiblist}

This command prints a bibliography list. In \biblatexsbl all abbreviations
(journals, series, and shorthands) can be printed using the following command:

\begin{quote}
  \verb+\printbiblist[...]{abbreviations}+
\end{quote}

See the \biblatex manual for an explanation of available optional arguments.

\cmditem{printbiblioraphy}

Inserts the bibliography. See the \biblatex manual for an explanation of
available optional arguments.

\end{ltxsyntax}

\subsection{Package Options}

\biblatexsbl defaults to the recommendations of the \citeshorthand{SBL}, but
it also supports many of the standard options from \biblatex as well as a few
custom ones outlined below.

\begin{optionlist}

\optitem[sbl]{citepages}{\opt{sbl}, \opt{permit}, \opt{omit}, \opt{separate}}

Use this option to fine-tune the formatting of the \bibfield{pages} field
the first time an entry is cited.

\begin{valuelist}
\item[sbl] The \bibfield{postnote} field is not printed for first citations.
  e.g.,

  \begin{snugshade}
    \samplecite{1}[159]{leyerle:1993}
    \samplecite{2}[159]{leyerle:1993}
    \samplecite{3}{leyerle:1993}
  \end{snugshade}

  If \bibfield{postnote} is not a page range, then it is printed in
  parentheses after \bibfield{pages}. e.g.,

  \begin{snugshade}
    \samplecite{1}[a note]{irvine:2014}
    \samplecite{2}[a note]{irvine:2014}
  \end{snugshade}

\item[permit] The \bibfield{postnote} is printed in parentheses after the
  \bibfield{pages} field. e.g.,

  \makeatletter\cbx@opt@citepages@permit\makeatother

  \begin{snugshade}
    \samplecite{1}[245]{wildberger:1965}
    \samplecite{2}[245]{wildberger:1965}
  \end{snugshade}

  If \bibfield{postnote} is not a page range, then \bibfield{pages} is printed
  for subsequent citations, and the \bibfield{postnote} is printed in
  parentheses. e.g.,

  \begin{snugshade}
    \samplecite{3}[a note]{wildberger:1965}
  \end{snugshade}

\item[omit] The \bibfield{pages} field is not printed unless
  \bibfield{postnote} is empty or not a page range. e.g.,

  \makeatletter\cbx@opt@citepages@omit\makeatother

  \begin{snugshade}
    \samplecite{1}[5]{freedman:1977}
    \samplecite{2}[5]{freedman:1977}
    \samplecite{3}{freedman:1977}
    \samplecite{4}[a note]{freedman:1977}
  \end{snugshade}

\item[separate] The \bibfield{postnote} is printed in parentheses after the
  \bibfield{pages} field preceeded by the bibliography string \sty{thiscite}.
  e.g.,

  \makeatletter\cbx@opt@citepages@separate\makeatother

  \begin{snugshade}
    \samplecite{1}[1]{petersen:1988}
    \samplecite{2}[1]{petersen:1988}
  \end{snugshade}

  If \bibfield{postnote} is not a page range, then \sty{firstcite} is not
  used. e.g.,

  \begin{snugshade}
    \samplecite{3}[a note]{leyerle:1993}
  \end{snugshade}

  \makeatletter\cbx@opt@citepages@sbl\makeatother

\end{valuelist}

\boolitem[false]{fullbibrefs}

The \emph{Student Supplement for the} \cite{SBLHS} permits two styles for the
bibliography entry for Bible dictionaries and encyclopaedias, and multivolume
commentaries for the entire Bible by multiple
authors.\autocite[4–5]{SBLHS:studentsupp}

This option applies to \bibfield{@inreference} and \bibfield{@incommentary}
entry types.

\begin{valuelist}
\item[true] The bibliography entry is printed in long form. e.g.,

  \begin{snugshade}
    \toggletrue{fullbibrefs}
    \nocite{IDB}
    \samplebib*{stendahl:1962}
    \togglefalse{fullbibrefs}
  \end{snugshade}

\item[false] The bibliography entry is printed in a short form. e.g.,

  \begin{snugshade}
    \samplebib*{stendahl:1962}
  \end{snugshade}
\end{valuelist}

\optitem[false]{ibidtracker}{\opt{true}, \opt{false}, \opt{context},
\opt{strict}, \opt{constrict}}

This option controls the \emph{ibidem} tracker. The possible choices are:

\begin{valuelist}
\item[true] Enable the tracker in global mode.
  not tracked separately between text body and footnotes.
\item[false] Disable the tracker: \emph{ibid.}\ will not be used.
\item[context] Enable the tracker in context-sensitive mode. In this mode,
  citations in footnotes and in the body text are tracked separately.
\item[strict] Enable the tracker in strict mode. In this mode, potentially
  ambiguous references are suppressed. A reference is considered ambiguous if
  either the current citation (the one including the \emph{ibidem}) or the
  previous citation (the one the \emph{ibidem} refers to) consists of a list
  of references.
\item[constrict] This mode combines the features of \opt{context} and
  \opt{strict}. It also keeps track of footnote numbers and detects
  potentially ambiguous references in footnotes in a stricter way than the
  \opt{strict} option. In addition to the conditions imposed by the
  \opt{strict} option, a reference in a footnote will only be considered as
  unambiguous if the current citation and the previous citation are given in
  the same footnote or in immediately consecutive footnotes.
\end{valuelist}

\boolitem[false]{ibidpage}

The scholarly abbreviation \emph{ibidem} is sometimes taken to mean both ‘same
author + same title’ and ‘same author + same title + same page’ in traditional
citation schemes. By default, this is not the case with this style because it
may lead to ambiguous citations. If you prefer the wider interpretation of
\emph{ibidem}, set the package option \opt{ibidpage=true} or simply
\opt{ibidpage} in the preamble. The default setting is \opt{ibidpage=false}.

\optitem[false]{idemtracker}{\opt{true}, \opt{false}, \opt{context},
\opt{strict}, \opt{constrict}}

This option controls the \emph{idem} tracker. The possible choices are:

\begin{valuelist}
\item[true] Enable the tracker in global mode.
\item[false] Disable the tracker: \emph{idem} will not be used.
\item[context] Enable the tracker in context-sensitive mode. In this mode,
  citations in footnotes and in the body text are tracked separately.
\item[strict] This is an alias for \opt{true}, provided only for consistency
  with the other trackers. Since \emph{idem} replacements do not get ambiguous
  in the same way as \emph{ibidem}, the strict tracking mode does not apply to
  them.
\item[constrict] This mode is similar to \opt{context} with one additional
  condition: a reference in a footnote will only be considered as unambiguous
  if the current citation and the previous citation are given in the same
  footnote.
\end{valuelist}

\optitem[false]{pagetracker}{\opt{true}, \opt{false}}

This option controls whether \emph{ibidem} and \emph{idem} are used across
page breaks or not.

\begin{valuelist}
\item[true] Enable the tracker in automatic mode. This is like \opt{spread} if
  LaTeX is in twoside mode, and like \opt{page} otherwise.
\item[false] Disable the tracker.
\item[page] Enable the tracker in page mode. In this mode, tracking works on a
  per-page basis.
\item[spread] Enable the tracker in spread mode. In this mode, tracking works
  on a per-spread (double page) basis.
\end{valuelist}

\optitem[comp]{releasedate}{\opt{year}, \opt{short}, \opt{long}, \opt{terse,
\opt{comp}, \opt{iso8601}}}

Similar to the \opt{date} option (see the \biblatex manual) but controls the
format of the \bibfield{releasedate}.

\boolitem[true]{sblfootnotes}

This option controls the style of footnotes. This option is compatible with
the \sty{footmisc} package provided \sty{footmisc} is loaded before \biblatex.

\begin{valuelist}
\item[true] Footnotes are printed with a normal number followed by a period
  and the first line indented:

  \begin{snugshade}
    \samplecite*{1}{talbert:1992}
  \end{snugshade}

\item[false] Footnotes are printed with a superscript (or whatever other
  default has been set up by your style):

  \begin{snugshade}
    \hspace*{\bibindent}\llap{\textsuperscript{1}}\cite{robinson+koester:1971}.
  \end{snugshade}
\end{valuelist}

\end{optionlist}

\section{Database Guide}

\subsection{Entry Types}

This section gives an overview of the entry types supported by \biblatexsbl.
Many work in the same way as \biblatex. Some standard entry types have custom
usage, and some are unique to \biblatexsbl. These are documented more fully.

\begin{typelist}

\typeitem{ancienttext}

This is a custom type for \biblatexsbl. It is used for the special examples in
\cite[§6.4.1, §6.4.3 and §6.4.8]{SBLHS}.

Unless \opt{options = \{skipbib=false\}} is set explicitly, a
\bibtype{ancienttext} entry will not appear in the bibliography. (Although,
see \opt{ANRW} \bibfield{entrysubtype} below for an exception.) The
\bibfield{sblxref} field is used to refer to the entry which should appear in
the bibliography instead of the \bibtype{ancienttext} entry.

The entry pointed to by \bibfield{sblxref} along with the \bibfield{postnote}
is printed in parentheses after the \bibfield{altpostnote}, \bibfield{editor},
and \bibfield{translator} fields they are present. \bibfield{translator} and
\bibfield{editor} fields are omitted for subsequent citations. e.g.,

\begin{snugshade}
  \samplecite{1}[319]{suppiluliumas}
  \samplecite*{2}[319]{suppiluliumas}
  \samplebib{ANET}
\end{snugshade}

If the entry contains \opt{options = \{skipbib=false\}}, then the bibliography
entry will be like \bibtype{book}. Any shorthand is also printed in the same
way as a \bibtype{book} shorthand.

The following values for the \bibfield{entrysubtype} field are supported:

\begin{valuelist}

\item[ANRW]

The \opt{ANRW} \bibfield{entrysubtype} is particularly for citing \cite{ANRW}
as outlined in §6.4.8 of the \cite{SBLHS}. In this case, the entry \emph{will}
appear in the bibliography. See \sty{biblatex-sbl-test.pdf} for full details
of the required database entry.

\item[chronicle]

Formats the \bibfield{title} using an upright shape font without quotation
marks. e.g.,

\begin{snugshade}
  \samplecite{1}[lines 3--4|125]{esarhaddonchronicle}
\end{snugshade}

\item[COS]

Suppresses parentheses around \emph{COS} and the \bibfield{postnote} for
subsequent citations. e.g.,

\begin{snugshade}
  \samplecite{1}{greathymnaten}
  \samplecite*{2}{greathymnaten}
\end{snugshade}

\item[inscription]

  Similarly to \opt{chronicle}, this formats the \bibfield{title} using an
  upright shape font without quotation marks.

\end{valuelist}

\typeitem{article}

An article in a journal or magazine. Also use this type for review articles
\parencite[§6.3.4]{SBLHS} and electronic journal articles
\parencite[§6.3.10]{SBLHS}.

\typeitem{book}

A single-volume book with one or more authors.

\typeitem{mvbook}

A multivolume \bibtype{book}. \biblatexsbl treats this as an alias for
\bibtype{book}.

There is one \bibfield{entrysubtype} supported:

\begin{valuelist}

\item[RIMA]

The citation for \citeseries{RIMA} \parencite[97]{SBLHS} is treated like a
series with a number when cited in full, but as a shorthand with a volume when
cited in short form. See \sty{biblatex-sbl-test.pdf} for full details.

\end{valuelist}

\typeitem{inbook}

A part of a book which forms a self-contained unit with its own title.
\biblatexsbl treats this as an alias for \bibtype{incollection}.

\typeitem{bookinbook}

This type is similar to \bibtype{inbook} but intended for works originally
published as a stand-alone book. The main difference is that the title is
printed in italics instead of in quotation marks. See §6.4.4 in
\sty{biblatex-sbl-test.pdf} for an example.

\typeitem{suppbook}

Supplemental material in a \bibtype{book}. Use this for an introduction,
preface or foreword written by someone other than the author
\parencite[§6.2.14]{SBLHS}. The \bibfield{type} field is used to specify the
type of supplementary material. See §6.2.14 of \sty{biblatex-sbl-test.pdf}.

\typeitem{classictext}

This type is a custom type for \biblatexsbl. It is used for the special
examples in \cite[§6.4.2 and §§6.4.4–6]{SBLHS}.

Unless \opt{options = \{skipbib=false\}} is set explicitly, a
\bibtype{classictext} entry will not appear in the bibliography. The
\bibfield{sblxref} field is used to refer to the entry which should appear in
the bibliography instead of the \bibtype{classictext} entry.

If present, the \bibfield{translator} and \bibfield{series} are printed in
parentheses following the \bibfield{postnote}. e.g.,

\begin{snugshade}
  \samplecite{1}[15.18-19]{tacitus:ann:jackson}
  \samplebib{tacitus}
\end{snugshade}
    
If the entry contains \opt{options = \{skipbib=false\}}, then the bibliography
entry will be like \bibtype{incollection} except that the \bibfield{title} is
set in italics instead of within quotation marks.

The following values for the \bibfield{entrysubtype} field are supported:

\begin{valuelist}

\item[churchfather]

Entries using the \opt{churchfather} \bibfield{entrysubtype} print the entry
pointed to by \bibfield{sblxref} within parentheses following the
\bibfield{altpostnote}. The \bibfield{postnote} field applies to the entry in
\bibfield{sblxref}. \bibfield{altpostnote} is always separated from the title
by a space.

\begin{snugshade}
  \samplecite{1}[28.3.5|252]{augustine:letters}
  \samplebib{augustine:letters}
\end{snugshade}

\end{valuelist}

\typeitem{collection}

A single-volume collection with multiple, self-contained contributions by
distinct authors which have their own title. The work as a whole has no
overall author but it will usually have an editor. \biblatexsbl treats this as
an alias for \bibtype{book}.

\typeitem{mvcollection}

A multi-volume \bibtype{collection}. \biblatexsbl treats this as an alias for
\bibtype{mvbook}.

\typeitem{incollection}

A contribution to a collection which forms a self-contained unit with a
distinct author and title.

\typeitem{commentary}

A single-volume commentary on a book (or part of a book) of the Bible by one
or more authors. This entry type is similar to \bibtype{book}, except that any
\bibfield{volume} and \bibfield{maintitle} is only printed in the
bibliography, not the citation.

\typeitem{mvcommentary}

A multi-volume commentary on a single book of the Bible by one or more authors
or a multi-volume commentary on the whole Bible by multiple authors.
\biblatexsbl treats this as an alias to \bibtype{mvbook}.

\typeitem{incommentary}

A contribution to a commentary which forms a self-contained unit with a
distinct author and title. This is typically a commentary on a book of the
Bible appearing in a single or multi-volume commentary on the entire Bible.

If an entry contains an \bibfield{sblxref} field, then the bibliography entry
is printed in either short or long form as described above under
\opt{fullbibrefs}. Otherwise this entry is treated as an alias for
\bibtype{incollection}.

\typeitem{conferencepaper}

An unpublished paper presented at a professional society. Use the
\bibfield{eventtitle}, \bibfield{venue}, and \bibfield{eventdate} fields to
specify relevant detail for the conference.

\typeitem{lexicon}

A single-volume lexicon or theological dictionary. \biblatexsbl treats this as
an alias for \bibtype{book}.

\typeitem{mvlexicon}

A multi-volume lexicon or theological dictionary. \biblatex treats this as an
alias for \bibtype{mvbook}.

\typeitem{inlexicon}

An article in a lexicon or theological dictionary. This is a custom type for
\biblatexsbl. The required \bibfield{sblxref} must contain the entry name of a
\bibtype{lexicon} or \bibtype{mvlexicon}. The \bibtype{inlexicon} entry does
not appear in the bibliography. Instead the lexicon pointed to by
\bibfield{sblxref} appears in the bibliography.

Subsequent citations do not include the article title, only the name of the
lexicon. e.g.,

\begin{snugshade}
  \samplecite{1}[511]{dahn+liefeld:see+vision+eye}
  \samplecite{2}[511]{dahn+liefeld:see+vision+eye}
  \samplebib{NIDNTT}
\end{snugshade}

\typeitem{manual}

Technical or other documentation, not necessarily in printed form. \biblatexsbl
treats this as an alias for \bibtype{book}.

\typeitem{misc}

A fallback type for entries which do not fit into any other category. Use the
\bibfield{howpublished} field to supply publishing information in free format,
if applicable.

This type is also set up to be able to insert an item into the list of
abbreviations. So it is permissible to include just \bibfield{journaltitle}
and \bibfield{shortjournal} or \bibfield{series} and \bibfield{shortseries},
or \bibfield{shorthand} and relevant fields. In this case ensure that
\bibfield{options = \{skipbib\}} is set.

\typeitem{online}

An online resource without a print counterpart. \biblatexsbl treats this as an
alias for \bibtype{article}.

\typeitem{proceedings}

A single-volume conference proceedings. In \biblatexsbl this as an alias for
\bibtype{collection}.

\typeitem{mvproceedings}

A multi-volume \bibtype{proceedings} entry. In \biblatexsbl this as an alias
for \bibtype{mvcollection}.

\typeitem{inproceedings}

An article in a conference proceedings. In \biblatexsbl this as an alias for
\bibtype{incollection}.

\typeitem{reference}

A single-volume encyclopaedia or dictionary. \biblatexsbl treats this as an
alias for \bibtype{book}.

\typeitem{mvreference}

A multi-volume \bibtype{reference}. \biblatexsbl treats this as an alias for
\bibtype{mvbook}.

\typeitem{inreference}

An article in an encyclopaedia or dictionary. The required \bibfield{sblxref}
field must contain the entry name of a \bibtype{reference} or
\bibtype{mvreference}.

The bibliography entry is printed in either short or long form as described
above under \opt{fullbibrefs}.

\typeitem{review}

A book review in a journal. This is similar to the \bibtype{article} entry
type. Use the \bibfield{revdauthor}\slash\bibfield{revdeditor} and
\bibfield{revdtitle} fields to specify the author\slash editor and title of
the book being reviewed.

Note that review articles are treated like articles and should use the
\bibtype{article} entry type.

\typeitem{seminarpaper}

An \citeshorthand{SBL} seminar paper. This is a custom entry type for
\biblatexsbl. See §6.4.11 of \sty{biblatex-sbl-test} for an example.

\typeitem{set}

An entry set. This entry type is special. See the \biblatex manual for
details.

\typeitem{thesis}

A unpublishes thesis written for an educational institution to satisfy the
requirements for a degree. Use the \bibfield{type} field to specify the type of
thesis and the \bibfield{institution} to specify the educational institution.

\typeitem{unpublished}

A work with an author and a title which has not been formally published, such
as a manuscript or the script of a talk. Use the fields howpublished and note
to supply additional information in free format, if applicable.

See §6.3.8 of \sty{biblatex-sbl-test} for an example.

\end{typelist}

\subsection{Entry Fields}

\biblatexsbl supports many of the entry fields outlined in the \biblatex
manual. There are also a number of custom entry fields supported by
\biblatexsbl. These are documented below.

\begin{fieldlist}

\listitem{bookeditor}{name}

The editor(s) of the \bibfield{booktitle}.

\listitem{booktranslator}{name}

The translator(s) of the \bibfield{booktitle}.

\listitem{maineditor}{name}

The editor(s) of the \bibfield{maintitle}.

\listitem{maintranslator}{name}

The translator(s) of the \bibfield{maintitle}.

\fielditem{releasedate}{date}

The date a text edition published online with no print counterpart is
released. See §6.4.13 of \sty{biblatex-sbl-test.pdf}.

\fielditem{releaseday}{datepart}

This field holds the day component of the \bibfield{releasedate} field.

\fielditem{releasemonth}{datepart}

This field holds the month component of the \bibfield{releasedate} field.

\fielditem{releaseyear}{datepart}

This field holds the year component of the \bibfield{releasedate} field.

\listitem{revdauthor}{name}

The author(s) of the \bibfield{revdtitle}.

\listitem{revdeditor}{name}

The editor(s) of the \bibfield{revdtitle}.

\fielditem{revdshorttitle}{literal}

The title of a book being review in an abridged form. This field is used in
subsequent citations of \bibtype{review} entry types.

\fielditem{revdsubtitle}{literal}

The subtitle of a book being reviewed.

\fielditem{revdtitle}{literal}

The title of a book being reviewed.

\fielditem{revdtitleaddon}{literal}

An annex to the \bibfield{revdtitle}, to be printed in a different font.

\listitem{revdtranslator}{name}

The translator(s) of the \bibfield{revdtitle}.

\fielditem{sblxref}{entry key}

This is a special cross-reference field. Unlike \bibfield{crossref} and
\bibfield{xref}, the parent entry will \emph{always} appear in the bibliography
and (if applicable) the list of abbreviations regardless of the value of
\opt{mincrossrefs}. Neither does the child entry inherit any fields from the
parent.

It is used when what appears in the citation is radically different to what
appears in the bibliography.

It's appearance in the bibliography and list of abbreviations can be
controlled using the entry options \opt{skipbib}, \opt{skipbiblist},
\opt{skipbiblistshorthand}, and \opt{skipbiblistseries}. See below for details
of these options.

\fielditem{seriesseries}{literal}

This field is used when a \bibfield{series} is begun anew to distinguish
between the old and new series. See \cite[§6.2.24]{SBLHS}.

\fielditem{shortbooktitle}{literal}

The \bibfield{booktitle} in abridged form.

\fielditem{shortmaintitle}{literal}

The \bibfield{maintitle} in abridged form.

\listitem{withauthor}{name}

The author(s) who assist the \bibfield{author}. See \bibfield{witheditortype},
below, for an example.

\fielditem{withauthortype}{literal}

The type of \bibfield{withauthor}. This field will affect the string used to
introduce the author(s) who assist the author. If unspecified, the
bibliography string \sty{with} is used.

\listitem{witheditor}{name}

The editor(s) who assist the \bibfield{editor}.

\fielditem{witheditortype}{literal}

The type of \bibfield{witheditor}. This field will affect the string used to
introduce the editor(s) who assist the editor. If unspecified, the
bibliography string \sty{with} is used.

\begin{snugshade}
  \samplecite{1}[1:24]{TLOT}
  \samplebib{TLOT}
  \samplebiblist{TLOT}
\end{snugshade}

\listitem{withtranslator}{name}

The translator(s) who assist the \bibfield{translator}.

\fielditem{withtranslatortype}{literal}

The type of \bibfield{withtranslator}. This field will affect the string used
to introduce the translator(s) who assist the translator. If unspecified, the
bibliography string \sty{with} is used.

\listitem{withbookauthor}{name}

The author(s) who assist the \bibfield{bookauthor}.

\fielditem{withbookauthortype}{literal}

This field is analogous to the \bibfield{withauthortype}, but for the
\bibfield{bookauthor}.

\listitem{withbookeditor}{name}

The editor(s) who assist the \bibfield{bookeditor}.

\fielditem{withbookeditortype}{literal}

This field is analogous to the \bibfield{witheditortype}, but for the
\bibfield{bookeditor}.

\listitem{withbooktranslator}{name}

The translator(s) who assist the \bibfield{booktranslator}.

\fielditem{withbooktranslatortype}{literal}

This field is analogous to the \bibfield{withtranslatortype}, but for the
\bibfield{booktranslator}.

\listitem{withmainauthor}{name}

The author(s) who assist the \bibfield{mainauthor}.

\fielditem{withmainauthortype}{literal}

This field is analogous to the \bibfield{withauthortype}, but for the
\bibfield{mainauthor}.

\listitem{withmaineditor}{name}

The editor(s) who assist the \bibfield{maineditor}.

\fielditem{withmaineditortype}{literal}

This field is analogous to the \bibfield{witheditortype}, but for the
\bibfield{maineditor}.

\listitem{withmaintranslator}{name}

The translator(s) who assist the \bibfield{maintranslator}.

\fielditem{withmaintranslatortype}{literal}

This field is analogous to the \bibfield{withtranslatortype}, but for the
\bibfield{maintranslator}.

\end{fieldlist}

\subsection{Entry Options}

\biblatexsbl supports many of the entry options outlined in the \biblatex
manual. There are also a number of custom entry options supported by
\biblatexsbl. These are documented below.

\begin{optionlist}

\optitem[false]{firstcitenoshorthand}{\opt{true}, \opt{false}}

This option controls the first citation of entries with a \bibfield{shorthand}
field. The possible choices are:

\begin{valuelist}
\item[true] Do not use the \bibfield{shorthand} the first time and entry is
  cited. The entry is cited in full as it would be if no \bibfield{shorthand}
  was present.
\item[false] Always use the \bibfield{shorthand} when citing the entry.
\end{valuelist}

\optitem[false]{nolongcite}{\opt{true}, \opt{false}}

This option controls the format of the first citation. The possible choices
are:

\begin{valuelist}
\item[true] Always use the short subsequent citation format, even the first
  time an entry is cited. In effect this sets \cmd{citeseen} to true for the
  first citation.
\item[false] Use a full citation the first time and entry is cited and a short
  citation for subsequent citations.
\end{valuelist}

\optitem[false]{shortciteauthor}{\opt{true}, \opt{false}}

This option controls the format of subsequent citations. The possible choices
are:

\begin{valuelist}
\item[true] Suppress the \bibfield{shorttitle} or \bibfield{title} in
  subsequent citations, so only the author(s) or editor(s) are printed.
\item[false] Include the \bibfield{shorttitle} or \bibfield{title} in
  subsequent citations.
\end{valuelist}

\optitem[false]{skipbiblistshorthand}{\opt{true}, \opt{false}}

This option controls what appears in the list of abbreviations for database
entries containing both a \bibfield{shorthand} and a \bibfield{shortseries}.
For entries not containing a \bibfield{shortseries} just use the option
\opt{skipbiblist}. The possible options are:

\begin{valuelist}
\item[true] Do not include the \bibfield{shorthand} in the list of
  abbreviations.
\item[false] Include the \bibfield{shorthand} in the list of abbreviations.
\end{valuelist}

\optitem[false]{skipbiblistshortseries}{\opt{true}, \opt{false}}

This option controls what appears in the list of abbreviations for database
entries containing both a \bibfield{shorthand} and a \bibfield{shortseries}.
For entries not containing a \bibfield{shorthand} just use the option
\opt{skipbiblist}. The possible options are:

\begin{valuelist}
\item[true] Do not include the \bibfield{shortseries} in the list of
  abbreviations.
\item[false] Include the \bibfield{shortseries} in the list of abbreviations.
\end{valuelist}

\end{optionlist}

\subsection{Reprints}

\biblatexsbl supports three different ways of doing reprints with varying
complexity.

If only the original publisher, location, and/or year are required, then use
the fields \bibfield{origpublisher}, \bibfield{origlocation}, and
\bibfield{origdate}. e.g.,

\begin{quote}
\begin{lstlisting}{}
@book{vanseters:1997,
    author = {Van Seters, John},
    title = {In Search of History: Histeriography in the Ancient
               World and the Origins of Biblical History},
    origlocation = {New Haven},
    origpublisher = {Yale University Press},
    origdate = {1983},
    location = {Winona Lake, IN},
    publisher = {Eisenbrauns},
    date = {1997}
}
\end{lstlisting}
\begin{snugshade}
  \samplecite{1}[90]{vanseters:1997}
  \samplebib{vanseters:1997}
\end{snugshade}
\end{quote}

When extra information is required, use a related entry with
\bibfield{relatedtype = \{reprint\}}. A custom string can be specified instead
of “Repr.” using the optional \bibfield{relatedstring} field. In this case no
punctuation is inserted after the \bibfield{relatedstring}. You could think of
the default being \bibfield{relatedstring = \{\cmd{bibstring}\{reprint\},\}}.
e.g.,

\begin{quote}
\begin{lstlisting}{}
@mvbook{sasson:2000,
    editor = {Sasson, Jack M.},
    title = {Civilizations of the Ancient Near East},
    volumes = {4},
    location = {New York},
    publisher = {Scribner's Sons},
    year = {1995},
    related = {sasson:repr},
    relatedtype = {reprint}
}
@mvbook{sasson:repr,
    volumes = {4~vols.\ in 2},
    location = {Peabody, MA},
    publisher = {Hendrickson},
    date = {2000}
}
\end{lstlisting}
\begin{snugshade}
  \samplecite{1}[1:40]{sasson:2000}
  \samplebib{sasson:2000}
\end{snugshade}
\end{quote}

A full reprint history also uses the \bibfield{related} field, but with some
other \bibfield{relatedtype} apart from \bibfield{relatedtype = \{reprint\}}.
e.g.,

\begin{quote}
\begin{lstlisting}{}
@book{wellhausen:1883,
    author = {Wellhausen, Julius},
    title = {Prolegomena zur Geschichte Israels},
    edition = {2},
    location = {Berlin},
    publisher = {Reimer},
    date = {1883}
}
@book{wellhausen:1885,
    author = {Wellhausen, Julius},
    title = {Prolegomena to the History of Israel},
    translator = {Black, J. Sutherland and Enzies, A.},
    preface = {Smith, W. Robertson},
    location = {Edinburgh},
    publisher = {Black},
    related = {wellhausen:1883},
    relatedtype = {translationof},
    date = {1885}
}
@book{wellhausen:1957,
    author = {Wellhausen, Julius},
    title = {Prolegomena to the History of Ancient Israel},
    location = {New York},
    publisher = {Meridian Books},
    related = {wellhausen:1885},
    relatedtype = {reprintof},
    date = {1957}
}
\end{lstlisting}
\begin{snugshade}
  \samplecite{1}[20]{wellhausen:1957}
  \samplebib{wellhausen:1957}
\end{snugshade}
\end{quote}

\printbiblist[heading=biblistintoc]{abbreviations}

\printbibliography[heading=bibintoc]

\printindex

\end{document}
