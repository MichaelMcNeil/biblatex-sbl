\documentclass[a4paper]{article}

\usepackage{fontspec}
\usepackage{microtype}
\usepackage{verbatim}
\usepackage{xcolor}
\usepackage[style=sbl,backend=biber]{biblatex}
\addbibresource{sbl.bib}

\newfontfamily\greekfont[Script=Greek,Contextuals=Alternate,Ligatures=Required]{SBL_BLit.ttf}
\newfontfamily\hebrewfont[Script=Hebrew,Contextuals=Alternate,Ligatures=Required]{SBL_BLit.ttf}
\newfontfamily\translitfont[Script=Latin]{SBL_BLit.ttf}
\newcommand{\gr}[1]{{\greekfont #1}}
\newcommand{\he}[1]{{\hebrewfont #1}}
\newcommand{\tl}[1]{{\translitfont #1}}

\setmonofont[Scale=MatchLowercase]{DejaVuSansMono.ttf}

\setlength{\parindent}{0em}
\setlength{\parskip}{0.5\baselineskip}

\definecolor{biblatex-colour}{rgb}{0.4,0.6,0.8}
\definecolor{reference-colour}{rgb}{1,0.6,0}

\newcommand\citetest[5]{%
  {\textbf{BibLaTeX Output}\par
   \nobreak
   \texttt{\textbackslash autocite[#2]\{#5\}}\par
   \color{biblatex-colour}
   #1. \cite[#2]{#5}.\par
   \color{black}
   \texttt{\textbackslash autocite[#4]\{#5\}}\par
   \color{biblatex-colour}
   #3. \cite[#4]{#5}.\par
   \hangindent\bibindent\bibentrycite{#5}.\par}}
\newcommand\citetestnb[5]{%
  {\textbf{BibLaTeX Output}\par
   \nobreak
   \texttt{\textbackslash autocite[#2]\{#5\}}\par
   \color{biblatex-colour}
   #1. \cite[#2]{#5}.\par
   \color{black}
   \texttt{\textbackslash autocite[#2]\{#5\}}\par
   \color{biblatex-colour}
   #3. \cite[#4]{#5}.}}
\newcommand\citetestns[3]{%
  {\textbf{BibLaTeX Output}\par
   \nobreak
   \texttt{\textbackslash autocite[#2]\{#3\}}\par
   \color{biblatex-colour}
   #1. \cite[#2]{#3}.\par
   \hangindent\bibindent\bibentrycite{#3}.\par}}
\newcommand\citetestnp[3]{%
  {\textbf{BibLaTeX Output}\par
   \nobreak
   \texttt{\textbackslash autocite\{#3\}}\par
   \color{biblatex-colour}
   #1. \cite{#3}.\par
   #2. \cite{#3}.\par
   \hangindent\bibindent\bibentrycite{#3}.\par}}
\newcommand\citetestnpf[4]{%
  {\textbf{BibLaTeX Output}\par
   \nobreak
   \texttt{\textbackslash autocite\{#4\}}\par
   \color{biblatex-colour}
   #1. \cite{#4}.\par
   \color{black}
   \texttt{\textbackslash cite[#3]\{#4\}}\par
   \color{biblatex-colour}
   #2. \cite[#3]{#4}.\par
   \hangindent\bibindent\bibentrycite{#4}.\par}}
\newcommand\citetestnsnp[2]{%
  {\textbf{BibLaTeX Output}\par
   \nobreak
   \texttt{\textbackslash autocite\{#2\}}\par
   \color{biblatex-colour}
   #1. \cite{#2}.\par
   \hangindent\bibindent\bibentrycite{#2}.\par}}
\newcommand\citetestnsnpnb[2]{%
  {\textbf{BibLaTeX Output}\par
   \nobreak
   \texttt{\textbackslash autocite\{#2\}}\par
   \color{biblatex-colour}
   #1. \cite{#2}.\par}}
\newcommand\citetestlex[2]{%
  {\textbf{BibLaTeX Output}\par
   \nobreak
   \texttt{\textbackslash autocite\{#2\}}\par
   \color{biblatex-colour}
   #1. \cite{#2}.\par}}
\newcommand\citetestlexns[3]{%
  {\textbf{BibLaTeX Output}\par
   \nobreak
   \texttt{\textbackslash autocite[#2]\{#3\}}\par
   \color{biblatex-colour}
   #1. \cite[#2]{#3}.\par}}
\newcommand\citetestbib[1]{%
  {\textbf{BibLaTeX Output}\par
   \nobreak
   \color{biblatex-colour}
   \hangindent\bibindent\bibentrycite{#1}.\par}}

\newenvironment{refimp}{%
  \begin{minipage}{\linewidth}
    \setlength{\parskip}{1ex}
    \textbf{Reference Implementation}\par
    \nobreak
    \color{reference-colour}
}{\end{minipage}}

\newenvironment{vb}{%
  \setlength{\parskip}{0pt}
  \verbatim}{\endverbatim}


\begin{document}

\setcounter{section}{5}

\section{General Examples}

The following examples define SBL style for notes and bibliographies more
fully than the select rules in §6 of the second edition of the SBL Handbook of
Style.

Note that the bib database entries shown below make extensive use of
abbreviations defined by \texttt{@string} in \texttt{sbl.bib}. See the top of
\texttt{sbl.bib} for the definitions used.

\setcounter{subsection}{1}
\subsection{General Examples: Books}

\subsubsection{A Book by a Single Author}

\begin{vb}
@book{talbert:1992,
  author = talbert,
  title = {Reading John: A Literary and Theological Commentary
           on the Fourth Gospel and the Johannine Epistles},
  location = newyork,
  publisher = crossroad,
  year = {1992}
}
\end{vb}

\citetest{15}{127}{19}{22}{talbert:1992}


\begin{refimp}
  15. Charles H. Talbert, \emph{Reading John: A Literary and Theological
  Commentary on the Fourth Gospel and the Johannine Epistles} (New York:
  Crossroad, 1992), 127.

  19. Talbert, \emph{Reading John,} 22.

  \hangindent\bibindent Talbert, Charles H. \emph{Reading John: A Literary and
  Theological Commentary on the Fourth Gospel and the Johannine Epistles.} New
  York: Crossroad, 1992.
\end{refimp}

\subsubsection{A Book by Two or Three Authors}

\begin{vb}
@book{robinson+koester:1971,
  author = robinson # " and " # koester,
  title = {Trajectories through Early Christianity},
  location = philadelphia,
  publisher = fortress,
  year = {1971}
}
\end{vb}  

\citetest{4}{237}{12}{23}{robinson+koester:1971}

\begin{refimp}
  4. James M. Robinson and Helmut Koester, \emph{Trajectories through Early
  Christianity} (Philadelphia: Fortress, 1971), 237.

  12. Robinson and Koester, \emph{Trajectories through Early Christianity,}
  23.

  \hangindent\bibindent Robinson, James M., and Helmut Koester. \emph{Trajectories
  through Early Christianity.} Philadelphia: Fortress, 1971.
\end{refimp}

\subsubsection{A Book by More than Three Authors}

\begin{vb}
@book{scott+etal:1993,
  author = scott # " and " # dean # " and " # sparks # " and "
           # lazar,
  title = {Reading New Testament Greek},
  location = peabody,
  publisher = hendrickson,
  year = {1993}
}
\end{vb}  

\citetest{7}{53}{9}{42}{scott+etal:1993}

\begin{refimp}
  7. Bernard Brandon Scott et al., \emph{Reading New Testament Greek} (Peabody, MA:
  Hendrickson, 1993), 53.
  
  9. Scott et al., \emph{Reading New Testament Greek,} 42.

  \hangindent\bibindent Scott, Bernard Brandon, Margaret Dean, Kristen Sparks,
  and Frances LaZar. \emph{Reading New Testament Greek.} Peabody, MA:
  Hendrickson, 1993.
\end{refimp}

\subsubsection{A Translated Volume}

\begin{vb}
@book{egger:1996,
  author = egger,
  title = {How to Read the New Testament: An Introduction to
           Linguistic and Historical-Critical Methodology},
  shorttitle = {How to Read},
  translator = heinegg,
  location = peabody,
  publisher = hendrickson,
  year = {1996}
}
\end{vb}  

\citetest{14}{28}{18}{291}{egger:1996}

\begin{refimp}
  14. Wilhelm Egger, \emph{How to Read the New Testament: An Introduction to
  Linguistic and Historical-Critical Methodology,} trans. Peter Heinegg
  (Peabody, MA: Hendrickson, 1996), 28.

  18. Egger, \emph{How to Read,} 291.
  
  \hangindent\bibindent Egger, Wilhelm. \emph{How to Read the New Testament:
  An Introduction to Linguistic and Historical-Critical Methodology.}
  Translated by Peter Heinegg. Peabody, MA: Hendrickson, 1996.
\end{refimp}

\subsubsection{The Full History of a Translated Volume}

TODO: Related entries not yet supported.

\begin{refimp}
  3. Julius Wellhausen, \emph{Prolegomena to the History of Ancient Israel}
  (New York: Meridian Books, 1957), 296; repr. of \emph{Prolegomena to the
  History of Israel,} trans.\ J. Sutherland Black and A. Enzies, with preface
  by W. Robertson Smith (Edinburgh: Black, 1885); trans.\ of \emph{Prolegomena
  zur Geschichte Israels,} 2nd ed, (Berlin: Reimer, 1883).

  \hangindent\bibindent Julius Wellhausen,\footnote{Should be “Wellhausen,
  Julius.”?} \emph{Prolegomena to the History of Ancient Israel.} New York:
  Meridian Books, 1957. Reprint of \emph{Prolegomena to the History of
  Israel.} Translated by J. Sutherland Black and A. Enzies, with preface by W.
  Robertson Smith. Edinburgh: Black, 1885. Translation of \emph{Prolegomena
  zur Geschichte Israels.} 2nd ed. Berlin: Reimer, 1883.
\end{refimp}

\subsubsection{A Book with One Editor}

\begin{vb}
@book{tigay:1985,
  editor = tigay,
  title = {Empirical Models for Biblical Criticism},
  shorttitle = {Empiracle Models},
  location = philadelphia,
  publisher = upp,
  year = {1985}
}
\end{vb}  

\citetest{5}{35}{9}{38}{tigay:1985}

\begin{refimp}
  5. Jeffrey H. Tigay, ed., \emph{Empirical Models for Biblical Criticism}
  (Philadelphia: University of Pennsylvania Press, 1985), 35.

  9. Tigay, \emph{Empirical Models,} 38.

  \hangindent\bibindent Tigay, Jeffrey H., ed. \emph{Empirical Models for
  Biblical Criticism.} Philadelphia: University of Pennsylvania Press,
  1985.
\end{refimp}

\subsubsection{A Book with Two or Three Editors}

\begin{vb}
@book{kaltner+mckenzie:2002,
  editor = kaltner # " and " # mckenzie,
  title = {Beyond Babel: A Handbook for Biblical Hebrew and
           Related Languages},
  series = RBS,
  shortseries = {RBS},
  number = {42},
  location = atlanta,
  publisher = sbl,
  year = {2002}
}
\end{vb}  

\citetest{44}{xii}{47}{viii}{kaltner+mckenzie:2002}

\begin{refimp}
  44. John Kaltner and Steven L. McKenzie, eds., \emph{Beyond Babel: A Handbook
  for Biblical Hebrew and Related Languages,} RBS 42 (Atlanta: Society of
  Biblical Literature, 2002), xii.

  47. Kaltner and McKenzie, viii.

  \hangindent\bibindent Kaltner, John, and Steven L. McKenzie, eds.
  \emph{Beyond Babel: A Handbook for Biblical Hebrew and Related Languages.}
  RBS 42. Atlanta: Society of Biblical Literature, 2002.
\end{refimp}

\subsubsection{A Book with Four or More Editors}

\begin{vb}
@book{oates+etal:2001,
  editor = oates # " and " # willis # " and " # bagnall #
           "and " # worp,
  title = {Checklist of Editions of Greek and Latin Papyri,
           Ostraca, and Tablets},
  edition = {5},
  series = BASPSup,
  shortseries = {BASPSup},
  number = {9},
  location = oakville,
  publisher = asp,
  year = {2001}
}
\end{vb}  

\citetestns{4}{10}{oates+etal:2001}

\begin{refimp}
  4. John F. Oates et al., eds., \emph{Checklist of Editions of Greek and Latin
  Papyri, Ostraca, and Tablets,} 5th~ed., BASPSup 9 (Oakville, CT: American
  Society of Papyrologists, 2001), 10.

  \hangindent\bibindent Oates, John F., William H. Willis, Roger S. Bagnall,
  and Klaas A. Worp, eds. \emph{Checklist of Editions of Greek and Latin
  Papyri, Ostraca, and Tablets.} 5th~ed. BASPSup 9. Oakville, CT: American
  Society of Papyrologists, 2001.
\end{refimp}

\subsubsection{A Book with Both Author and Editor}

\begin{vb}
@book{schillebeeckx:1986,
  author = schillebeeckx,
  title = {The Schillebeeckx Reader},
  editor = schreiter,
  location = edinburgh,
  publisher = ttclark,
  year = {1986}
}
\end{vb}  

\citetestns{45}{20}{schillebeeckx:1986}

\begin{refimp}
  45. Edward Schillebeeckx, \emph{The Schillebeeckx Reader,} ed. Robert J.
  Schreiter (Edinburgh: T\&T Clark, 1986), 20.

  \hangindent\bibindent Schillebeeckx, Edward. \emph{The Schillebeeckx
  Reader.} Edited by Robert J. Schreiter. Edinburgh: T\&T Clark, 1986.
\end{refimp}

\subsubsection{A Book with Author, Editor, and Translator}

\begin{vb}
@book{blass+debrunner:1982,
  author = blass # " and " # debrunner,
  title = {Grammatica del greco del Nuovo Testamento},
  editor = rehkopf,
  translator = pisi,
  location = brescia,
  publisher = paideia,
  year = {1982}
}
\end{vb}  

\citetestns{3}{40}{blass+debrunner:1982}

\begin{refimp}
  3. Friedrich Blass and Albert Debrunner, \emph{Grammatica del greco del
  Nuovo Testamento,} ed. Friedrich Rehkopf, trans. Giordana Pisi (Brescia:
  Paideia, 1982), 40.

  \hangindent\bibindent Blass, Friedrich, and Albert Debrunner.
  \emph{Grammatica del greco del Nuovo Testamento.} Edited by Friedrich
  Rehkopf. Translated by Giordana Pisi. Brescia: Paideia, 1982.
\end{refimp}

\subsubsection{A Title in a Modern Work Citing a Nonroman Alphabet}

\begin{vb}
@article{irvine:2014,
  author = irvine,
  title = {Idols [\emph{ktbwnm}]: A note on Hosea 13:2a},
  journal = JBL,
  shortjournal = {JBL},
  volume = {133},
  year = {2014},
  pages = {509-517}
}
\end{vb}

\citetestnsnpnb{34}{irvine:2014}

\begin{refimp}
  34. Stuart A. Irvine, “Idols [\emph{ktbwnm}]: A Note on Hosea 13:2a,”
  \emph{JBL} 133 (2014): 509–17.
\end{refimp}

\subsubsection{An Article in and Edited Volume}

\begin{vb}
@incollection{attridge:1986,
  author = attridge,
  title = {Jewish Historiography},
  pages = {311-343},
  booktitle = {Early Judaism and Its Modern Interpreters},
  editor = kraft # " and " # nickelsburg,
  publisher = philadelphia # ": " # fortress # "; " #
              atlanta # ": " # scholars,
  year = {1986}
}
\end{vb}  

\citetest{3}{311-343}{6}{314-317}{attridge:1986}

\begin{refimp}
  3. Harold W. Attridge, “Jewish Historiography,” in \emph{Early Judaism and
  Its Modern Interpreters,} ed. Robert A. Kraft and George W. E. Nickelsburg
  (Philadelphia: Fortress; Atlanta: Scholars Press, 1986), 311–43.
  
  6. Attridge, “Jewish Historiography,” 314–17.

  Attridge, Harold A. “Jewish Historiography.” Pages 311–43 in \emph{Early
  Judaism and Its Modern Interpreters.} Edited by Robert A. Kraft and George
  W. E. Nickelsburg. Philadelphia: Fortress; Atlanta: Scholars Press, 1986.
\end{refimp}

\subsubsection{An Article in a Festschrift}

\begin{vb}
@incollection{vanseters:1995,
  author = vanseters,
  title = {The Theology of the Yahwist: A Preliminary Sketch},
  shortttitle = {Theology of the Yahwist},
  pages = {219-228},
  booktitle = {“Wer ist wie du, Herr, unter den Göttern?”:
               Studien zur Theologie und Religionsgeschichte
               Israels für Otto Kaiser zum 70. Geburtstag},
  editor = kottsieper # " and others",
  location = göttingen,
  publisher = vandr,
  year = {1995}
}
\end{vb}  

\citetest{8}{219-228}{17}{222}{vanseters:1995}

\begin{refimp}
  8. John Van Seters, “The Theology of the Yahwist: A Preliminary Sketch,” in
  \emph{“Wer ist wie du, Herr, unter den Göttern?”: Studien zur Theologie und
  Religionsgeschichte Israels für Otto Kaiser zum 70. Geburtstag,} ed. Ingo
  Kottsieper et al. (Göttingen: Vandenhoeck \& Ruprecht, 1995), 219–28.

  17. Van Seters, “Theology of the Yahwist,” 222.

  \hangindent\bibindent Van Seters, John. “The Theology of the Yahwist: A
  Preliminary Sketch.” Pages~219–28 in \emph{“Wer ist wie du, Herr, unter den
  Göttern?”: Studien zur Theologie und Religionsgeschichte Israels für Otto
  Kaiser zum 70. Geburtstag.} Edited by Ingo Kottsieper et al. Göttingen:
  Vandenhoeck \& Ruprecht, 1995.
\end{refimp}

\subsubsection{An Introduction, Preface, or Foreword Written by Someone Other
Than the Author}

\begin{vb}
@suppbook{boers:1996,
  author = boers,
  type = {introduction},
  booktitle = {How to Read the New Testament: An Introduction to
               Linguistic and Historical-Critical Methodology},
  bookauthor = egger,
  translator = heinegg,
  location = peabody,
  publisher = hendrickson,
  year = {1996}
}
\end{vb}  

\citetest{2}{xi-xxi}{6}{xi-xx}{boers:1996}

\begin{refimp}
  2. Hendrikus Boers, introduction to \emph{How to Read the New Testament: An
  Introduction to Linguistic and Historical-Critical Methodology,} by Wilhelm
  Egger, trans. Peter Heinegg (Peabody; MA: Hendrickson, 1996), xi-xxi.

  6. Boers, introduction, xi-xx.

  \hangindent\bibindent Boers, Hendrikus. Introduction to \emph{How to Read
  the New Testament: An Introduction to Linguistic and Historical-Critical
  Methodology,} by Wilhelm Egger. Translated by Peter Heinegg. Peabody; MA:
  Hendrickson, 1996.
\end{refimp}

\subsubsection{Multiple Publishers for a Single Book}

\begin{vb}
@book{gerhardsson:1961,
  author = gerhardsson,
  title = {Memory and Manuscript: Oral Tradition and Written
           Transmission in Rabbinic Judaism and Early Christianity},
  series = ASNU,
  shortseries = {ASNU},
  number = {22},
  publisher = lund # ": " # gleerup # "; " #
              copenhagen # ": " # munksgaard,
  year = {1961}
}
\end{vb}

\citetestbib{gerhardsson:1961}

\begin{refimp}
  \hangindent\bibindent Birger Gerhardsson,\footnote{Should be “Gerhardsson,
  Birger.”?} \emph{Memory and Manuscript: Oral Tradition and Written
  Transmission in Rabbinic Judaism and Early Christianity.} ASNU 22. Lund:
  Gleerup; Copenhagen: Munksgaard, 1961.
\end{refimp}

\subsubsection{A Revised Edition}

\begin{vb}
@book{pritchard:1969,
  editor = pritchard,
  title = {Ancient Near Eastern Texts Relating to the Old
           Testament},
  edition = {3},
  location = princeton,
  publisher = pup,
  year = {1969}
}
\end{vb}  

\citetestns{87}{xxi}{pritchard:1969}

\begin{refimp}
  87. James B. Pritchard, ed., \emph{Ancient Near Eastern Texts Relating to
  the Old Testament,} 3rd ed. (Princeton: Princeton University Press, 1969),
  xxi.

  Pritchard, James B., ed. \emph{Ancient Near Eastern Texts Relating to the
  Old Testament.} 3rd ed. Princeton: Princeton University Press, 1969.
\end{refimp}

\vspace{\parskip}
\begin{vb}
@book{blenkinsopp:1996,
  author = blenkinsopp,
  title = {A History of Prophecy in Israel},
  edition = {\autocap{r}ev.\ and enl.\ ed.},
  location = louisville,
  publisher = wjk,
  year = {1996}
} 
\end{vb}  

\citetestns{56}{81}{blenkinsopp:1996}

\begin{refimp}
  56. Joseph Blenkinsopp, \emph{A History of Prophecy in Israel,} rev.\ and
  enl.\ ed. (Louisville: Westminster John Knox, 1996), 81.

  Blenkinsopp, Joseph. \emph{A History of Prophecy in Israel.} Rev.\ and enl.\
  ed. Louisville: Westminster John Knox, 1996.
\end{refimp}

\subsubsection{A Recent Reprint Title}

\begin{vb}
@book{vanseters:1997,
  author = vanseters,
  title = {In Search of History: Histeriography in the Ancient
           World and the Origins of Biblical History},
  origlocation = newhavan,
  origpublisher = yup,
  origyear = {1983},
  location = winonalake,
  publisher = eisenbrauns,
  year = {1997}
}
\end{vb}  

\citetestns{5}{35}{vanseters:1997}

\begin{refimp}
  5. John Van Seters, \emph{In Search of History: Historiography in the
  Ancient World and the Origins of Biblical History} (New Haven: Yale
  University Press, 1983; repr., Winona Lake, IN: Eisenbrauns, 1997), 35.

  \hangindent\bibindent Van Seters, John. \emph{In Search of History:
  Historiography in the Ancient World and the Origins of Biblical History.}
  New Haven: Yale University Press, 1983. Repr., Winona Lake, IN: Eisenbrauns,
  1997.
\end{refimp}

\subsubsection{A Reprint Title in the Public Domain}

\begin{vb}
@book{deissmann:1995,
  author = deissmann,
  title = {Light from the Ancient East: The New Testament
           Illustrated by Recently Discovered Texts of the
           Graeco-Roman World},
  origtranslator = strachan,
  origyear = {1927},
  location = peabody,
  publisher = hendrickson,
  year = {1955}
}
\end{vb}  

\citetestns{5}{55}{deissmann:1995}

\begin{refimp}
  5. Gustav Adolf Deissmann, \emph{Light from the Ancient East: The New Testament
  Illustrated by Recently Discovered Texts of the Graeco-Roman World} (trans.
  Lionel R. M. Strachan; 1927; repr., Peabody, MA: Hendrickson, 1995), 55.

  \hangindent\bibindent Deissmann, Gustav Adolf. \emph{Light from the Ancient
  East: The New Testament Illustrated by Recently Discovered Texts of the
  Graeco-Roman World.} Translated by Lionel R. M. Strachan. 1927. Repr., Peabody,
  MA: Hendrickson, 1995.
\end{refimp}

\subsubsection{A Forthcoming Book}

\begin{vb}
@book{harrison+welborn:forthcoming,
  author = harrison # " and " # welborn,
  title = {The First Urban Churches 2: Roman Corinth},
  shorttitle = {Roman Corinth},
  series = WGRWSup,
  shortseries = {WGRWSup},
  location = atlanta,
  publisher = sblpress,
  year = {forthcoming}
}
\end{vb}

\citetestnpf{9}{12}{201}{harrison+welborn:forthcoming}

\begin{refimp}
  9. James R. Harrison and L. L. Welborn, eds., \emph{The First Urban Churches
  2: Roman Corinth,} WGRWSup (Atlanta: SBL Press, forthcoming).

  12. Harrison and Welborn, \emph{Roman Corinth,} 201.

  \hangindent\bibindent Harrison, James R.\footnote{Should be “Harrison, James
  R.,”} and L. L. Welborn, eds. \emph{The First Urban Churches 2: Roman
  Corinth.} WGRWSup. Atlanta: SBL Press, forthcoming.
\end{refimp}

\subsubsection{A Multivolume Work}

\begin{vb}
@mvbook{harnack:1896-1905,
  author = harnack,
  title = {History of Dogma},
  translator = buchanan,
  origlanguage = {the 3rd German ed.},
  volumes = {7},
  location = boston,
  publisher = littlebrown,
  year = {1896-1905}
}
\end{vb}  

\citetest{5}{56}{9}{2:126}{harnack:1896-1905}

\begin{refimp}
  5. Adolf Harnack, \emph{History of Dogma,} trans. Neil Buchanan, 7 vols.
  (Boston: Little, Brown), 56.

  9. Harnack, \emph{History of Dogma,} 2:126.

  \hangindent\bibindent Harnack, Adolf. \emph{History of Dogma.} Translated
  from the 3rd German ed. by Neil Buchanan. 7 vols. Boston: Little, Brown,
  1896-1905.
\end{refimp}

\subsubsection{A Titled Volume in a Multivolume, Edited Work}

\begin{vb}
@collection{winter+clarke:1993,
  editor = winter # " and " # clarke,
  title = {The Book of Acts in Its Ancient Literary Setting},
  shorttitle = {Book of Acts},
  volume = {1},
  maintitle = {The Book of Acts in Its First Century Setting},
  maineditor = winter,
  location = grandrapids,
  publisher = eerdmans,
  year = {1993}
}
\end{vb}  

\citetest{5}{25}{16}{25}{winter+clarke:1993}

\begin{refimp}
  5. Bruce W. Winter and Andrew D. Clarke, eds., \emph{The Book of Acts in Its
  Ancient Literary Setting,} vol.~1 of \emph{The Book of Acts in Its First
  Century Setting,} ed. Bruce W. Winter (Grand Rapids: Eerdmans, 1993), 25.

  16. Winter and Clarke, \emph{Book of Acts,} 25.

  \hangindent\bibindent Winter, Bruce W., and Andrew D. Clarke, eds. \emph{The
  Book of Acts in Its Ancient Literary Setting.} Vol.~1 of \emph{The Book of
  Acts in Its First Century Setting.} Edited by Bruce W. Winter. Grand Rapids:
  Eerdmans, 1993.
\end{refimp}

\subsubsection{A Chapter within a Multivolume Work}

\begin{vb}
@incollection{mason:1996,
  author = mason,
  title = {Josephus on Canon and Scriptures},
  pages = {217-235},
  volume = {1},
  part = {1},
  maintitle = {Hebrew Bible\slash Old Testament: The History of
               Its Interpretation},
  editor = sæbø,
  location = göttingen,
  publisher = vandr,
  year = {1996}
}
\end{vb}

\citetest{24}{217-235}{28}{224}{mason:1996}

\begin{refimp}
  24. Steve Mason, “Josephus on Canon and Scriptures,” in \emph{Hebrew
  Bible\slash Old Testament: The History of Its Interpretation,} ed. Magne
  Saebø (Göttingen: Vandenhoeck \& Ruprecht, 1996),
  1.1:217–335.\footnote{Should be “1.1:217–35.”?}

  28. Mason, “Josephus on Canon and Scriptures,” 224.

  \hangindent\bibindent Mason, Steve. “Josephus on Canon and Scriptures.”
  Pages 217–35 in vol.~1, part 1 of \emph{Hebrew Bible\slash Old Testament:
  The History of Its Interpretation.} Edited by Magne Saebø. Göttingen:
  Vandenhoeck \& Ruprecht, 1996.
\end{refimp}

\subsubsection{A Chapter within a Titled Volume in a Multivolume Work}

\begin{vb}
@incollection{peterson:1993,
  author = peterson,
  title = {The Motif of Fulfilment and the Purpose of Luke-Acts},
  shorttitle = {Motif of Fulfilment},
  pages = {83-104},
  booktitle = {The Book of Acts in Its Ancient Literary Setting},
  bookeditor = winter # " and " # clarke,
  volume = {1},
  maintitle = {The Book of Acts in Its First Century Setting},
  maineditor = winter,
  location = grandrapids,
  publisher = eerdmans,
  year = {1993}
}
\end{vb}  

\citetest{66}{83-104}{78}{92}{peterson:1993}

\begin{refimp}
  66. David Peterson, “The Motif of Fulfilment and the Purpose of Luke-Acts,”
  in \emph{The Book of Acts in Its Ancient Literary Setting,} ed. Bruce W.
  Winter and Andrew D. Clarke, vol.~1 of \emph{The Book of Acts in Its First
  Century Setting,} ed. Bruce W. Winter (Grand Rapids: Eerdmans, 1993),
  83–104.

  78. Peterson, “Motif of Fulfilment,” 92.

  \hangindent\bibindent David Peterson,\footnote{Should be “Peterson,
  David.”?} “The Motif of Fulfilment and the Purpose of Luke-Acts.” Pages
  83–104 in \emph{The Book of Acts in Its Ancient Literary Setting.} Edited by
  Bruce W. Winter and Andrew D. Clarke. Vol.~1 of \emph{The Book of Acts in
  Its First Century Setting.} Edited by Bruce W. Winter. Grand Rapids:
  Eerdmans, 1993.
\end{refimp}

\subsubsection{A Work in a Series}

\begin{vb}
@book{hofius:1989,
  author = hofius,
  title = {Paulusstudien},
  series = WUNT,
  shortseries = {WUNT},
  number = {51},
  location = tübingen,
  publisher = mohrsiebeck,
  year = {1989}
}
\end{vb}  

\citetest{12}{122}{14}{124}{hofius:1989}

\begin{refimp}
  12. Otfried Hofius, \emph{Paulusstudien,} WUNT 51 (Tübingen: Mohr Siebeck,
  1989), 122.

  14. Hofius, \emph{Paulusstudien,} 124.

  Hofius, Otfried. \emph{Paulusstudien.} WUNT 51. Tübingen: Mohr Siebeck,
  1989.
\end{refimp}

\begin{vb}
@book{jeremias:1967,
  author = jeremias,
  title = {The Prayers of Jesus},
  shorttitle = {Prayers},
  series = SBT2,
  shortseries = {SBT},
  seriesseries = {2},
  number = {6},
  location = naperville,
  publisher = allenson,
  year = {1967}
}
\end{vb}

\citetest{23}{123-127}{32}{126}{jeremias:1967}

\begin{refimp}
  23. Joachim Jeremias, \emph{The Prayers of Jesus,} SBT 2/6 (Naperville, IL:
  Allenson, 1967), 123–27.

  32. Jeremias, \emph{Prayers,} 126.

  \hangindent\bibindent Jeremias, Joachim. The Prayers of Jesus. SBT 2/6.
  Naperville, IL: Allenson, 1967.
\end{refimp}

\subsubsection{Electronic Book}

\begin{vb}
@book{reventlow:2009,
  author = reventlow,
  title = {From the Old Testament to Origen},
  volume = {1},
  maintitle = {History of Biblical Interpretation},
  maintranslator = perdue,
  location = atlanta,
  publisher = sbl,
  year = 2009,
  eprint = nook,
  eprinttype = {ebook}
}
\end{vb}

\citetest{14}{ch.~1.3}{18}{ch.~1.3}{reventlow:2009}

\begin{refimp}
  14. Henning Graf Reventlow, \emph{From the Old Testament to Origen.}
  Vol.~1\footnote{Should be “, vol.~1”?} of \emph{History of Biblical
  Interpretation,} trans. Leo G. Perdue (Atlanta: Society of Biblical
  Literature, 2009), Nook edition, ch.~1.3.

  18. Reventlow, \emph{From the Old Testament to Origen,} ch.~1.3.
  
  \hangindent\bibindent Reventlow, Henning Graf. \emph{From the Old Testament
  to Origen.} Volume\footnote{Should be “Vol.” as elsewhere?} 1 of
  \emph{History of Biblical Interpretation.} Translated by Leo G. Perdue.
  Atlanta: Society of Biblical Literature, 2009. Nook edition.
\end{refimp}

\begin{vb}
@book{wright:2014,
  author = wright,
  title = {David, King of Israel, and Caleb in Biblical Memory},
  shorttitle = {David, King of Israel},
  location = cambridge,
  publisher = cup,
  date = {2014},
  eprint = kindle,
  eprinttype = {ebook}
}
\end{vb}

\citetest{3}{ch.~3, \mkbibquote{Introducing David}}{21}{ch.~5,
\mkbibquote{Evidence from Qumran}}{wright:2014}

\begin{refimp}
  3. Jacob L. Wright, \emph{David, King of Israel, and Caleb in Biblical
  Memory} (Cambridge: Cambridge University Press, 2014), Kindle edition,
  ch.~3, “Introducing David.”

  21. Wright, \emph{David, King of Israel,} ch.~5, “Evidence from Qumran.”

  \hangindent\bibindent Jacob L. Wright, \emph{David, King of Israel, and
  Caleb in Biblical Memory.} Cambridge: Cambridge University Press, 2014.
  Kindle edition.
\end{refimp}

\begin{vb}
@book{killebrew+steiner:2014,
  editor = killebrew # " and " # steiner,
  title = {The Oxford Handbook of the Archeology of
           the Levant: c.~8000–332 BCE},
  location = oxford,
  publisher = oup,
  date = {2014},
  doi = {10.1093/oxfordhb/9780199212972.001.0001}
}
\end{vb}

\citetestnp{53}{55}{killebrew+steiner:2014}

\begin{refimp}
  53. Ann E. Killebrew and Margreet Steiner, eds., \emph{The Oxford Handbook
  of the Archaeology of the Levant: c.~8000–332 BCE} (Oxford: Oxford
  University Press, 2014), doi:10.1093/oxfordhb/9780199212972.001.0001.

  55. Killebrew and Steiner, \emph{Archaeology of the Levant.}

  \hangindent\bibindent Killebrew, Ann E. and Margreet Steiner, eds. \emph{The
  Oxford Handbook of the Archaeology of the Levant: c.~8000–332 BCE.} Oxford:
  Oxford University Press, 2014. doi:10.1093/ oxfordhb/9780199212972.001.0001.
\end{refimp}

\begin{vb}
@book{kaufman:1974,
  author = kaufman,
  title = {The Akkadian Influences on Aramaic},
  series = AS,
  shortseries = {AS},
  number = {19},
  location = chicago,
  publisher = oiuc,
  year = {1974},
  url = {http://oi.uchicago.edu/pdf/as19.pdf}
}
\end{vb}

\citetestnpf{29}{32}{123}{kaufman:1974}

\begin{refimp}
  29. Stephen Kaufman. \emph{The Akkadian Influences on Aramaic,} AS 19
  (Chicago: The Oriental Institute of the University of Chicago, 1974),
  \url{http://oi.uchicago.edu/pdf/as19.pdf.}

  32. Kaufman, \emph{Akkadian Influences on Aramaic,} 123.

  \hangindent\bibindent Kaufman, Stephen. \emph{The Akkadian Influences on
  Aramaic.} AS 19. Chicago: The Oriental Institute of the University of
  Chicago, 1974. \url{http://oi.uchicago.edu/pdf/as19.pdf.}
\end{refimp}

\subsection{General Examples: Journal Articles, Reviews, and Dissertations}

\subsubsection{A Journal Article}

\begin{vb}
@article{leyerle:1993,
  author = leyerle,
  title = {John Chrysostom on the Gaze},
  shorttitle = {Chrysostom},
  journaltitle = JECS,
  shortjournal = {JECS},
  volume = {1},
  year = {1993},
  pages = {159-74}
}
\end{vb}  

\citetest{7}{159-174}{23}{161}{leyerle:1993}

\begin{refimp}
  7. Blake Leyerle, “John Chrysostom on the Gaze,” \emph{JECS} 1 (1993):159–74.

  23. Leyerle, “John Chrysostom,” 161.

  \hangindent\bibindent Leyerle, Blake. “John Chrysostom on the Gaze.”
  \emph{JECS} 1 (1993):159–74.
\end{refimp}

\subsubsection{A Journal Article with Multiple Page Locations and Volumes}

\begin{vb}
@article{wildberger:1965,
  author = wildberger,
  title = {Das Abbild Gottes: Gen 1:26–30},
  journal = TZ,
  shortjournal = {TZ},
  volume = {21},
  year = {1965},
  pages = {245-259, 481-501}
}
\end{vb}

\citetestns{21}{245–59, 481–501}{wildberger:1965}

\begin{refimp}
  21. Hans Wildberger, “Das Abbild Gottes: Gen 1:26–30,” \emph{TZ} 21 (1965):
  245–59, 481–501.

  \hangindent\bibindent Wildberger, Hans. “Das Abbild Gottes: Gen 1:26–30.”
  \emph{TZ} 21 (1965): 245–59, 481–501.
\end{refimp}

\begin{vb}
@article{wellhausen:1876-1877,
  author = wellhausen,
  title = {Die Composition des Hexateuchs},
  journal = JDT,
  shortjournal = {JDT},
  volume = {21 (1876): 39–450; 22},
  year = {1877},
  pages = {407-479}
}
\end{vb}

\citetestns{24}{21 (1876): 392–450; 22 (1877): 407–79}{wellhausen:1876-1877}

\begin{refimp}
  24. Julius Wellhausen, “Die Composition des Hexateuchs,” \emph{JDT} 21
  (1876): 392–450; 22 (1877): 407–79.

  \hangindent\bibindent Wellhausen, Julius. “Die Composition des Hexateuchs.”
  \emph{JDT} 21 (1876): 392–450; 22 (1877): 407–79.
\end{refimp}

\subsubsection{A Journal Article Republished in a Collected Volume}

\begin{vb}
@article{freedman:1977,
  author = freedman,
  title = {Pottery, Poetry, and Prophecy: An Essay on Biblical
           Poetry},
  journal = JBL,
  shortjournal = {JBL},
  volume = {96},
  year = {1977},
  pages = {5-26}
}
\end{vb}

\citetestns{20}{20}{freedman:1977}

\begin{refimp}
  20. David Noel Freedman, “Pottery, Poetry, and Prophecy: An Essay on
  Biblical Poetry,” \emph{JBL} 96 (1977): 20.\footnote{Should be “5–26” (full
    page reference) as elsewhere?}

  \hangindent\bibindent Freedman, David Noel. “Pottery, Poetry, and Prophecy:
  An Essay on Biblical Poetry.” \emph{JBL} 96 (1977): 5–26.
\end{refimp}

\begin{vb}
@incollection{freedman:1980,
  author = freedman,
  title = {Pottery, Poetry, and Prophecy: An Essay on Biblical
           Poetry},
  booktitle = {Pottery, Poetry, and Prophecy: Studies in Early
               Hebrew Poetry},
  location = winonalake,
  publisher = eisenbrauns,
  year = {1980},
  pages = {1-22}
}
\end{vb}

\citetestns{20}{14}{freedman:1980}

\begin{refimp}
  20. David Noel Freedman, “Pottery, Poetry, and Prophecy: An Essay on Biblical
  Poetry,” in \emph{Pottery, Poetry, and Prophecy: Studies in Early Hebrew
  Poetry} (Winona Lake, IN: Eisenbrauns, 1980), 14.\footnote{Should be “1–22”
  (full page reference) as elsewhere?}

  \hangindent\bibindent Freedman, David Noel. “Pottery, Poetry, and Prophecy:
  An Essay on Biblical Poetry.” Pages 1–22 in \emph{Pottery, Poetry, and
  Prophecy: Studies in Early Hebrew Poetry.} Winona Lake, IN: Eisenbrauns,
  1980.
\end{refimp}

\subsubsection{A Book Review}

\begin{vb}
@review{teeple:1966,
  author = teeple,
  revdauthor = robert # " and " # feuillet,
  revdtitle = {Introduction to the New Testament},
  journaltitle = JBR,
  shortjournaltitle = {JBR},
  volume = {34},
  year = {1966},
  pages = {368-370}
}
\end{vb}

\citetest{8}{368}{21}{369}{teeple:1966}

\begin{refimp}
  8. Howard M. Teeple, review of \emph{Introduction to the New Testament,} by
  André Robert and André Feuillet, \emph{JBR} 34 (1966): 368–70.
  
  21. Teeple, review of \emph{Introduction to the New Testament} (by Robert
  and Feuillet), 369.

  \hangindent\bibindent Teeple, Howard M. Review of \emph{Introduction to the
  New Testament,} by André Robert and André Feuillet. \emph{JBR} 34 (1966):
  368–70.
\end{refimp}

\begin{vb}
@review{pelikan:1992,
  author = pelikan,
  title = {The Things That You're Liable to Read in the Bible},
  shorttitle = {Things That You're Liable to Read},
  revdeditor = freedman,
  revdtitle = {The Anchor Bible Dictionary},
  journaltitle = {New York Times Review of Books},
  day = {20},
  month = {12},
  year = {1992},
  pages = {3}
}
\end{vb}

\citetestns{9}{3}{pelikan:1992}

\begin{refimp}
  9. Jaroslav Pelikan, “The Things That You're Liable to Read in the Bible,”
  review of \emph{The Anchor Bible Dictionary,} ed. David Noel Freedman.
  \emph{New York Times Review of Books,} 20 December 1992, 3.\footnote{Should
  be “(20 December 1992): 3.” as other journal examples?}
  
  Pelikan, Jaroslav. “The Things That You're Liable to Read in the Bible,”
  review of \emph{The Anchor Bible Dictionary,} ed. David Noel Freedman.
  \emph{New York Times Review of Books,} 20 December 1992,
  3.\textsuperscript{\emph{a}}
\end{refimp}

\begin{vb}
@article{petersen:1988,
  author = petersen,
  title = {Hebrew Bible Textbooks},
  subtitle = {A Review Article},
  journaltitle = CRBR,
  shortjournal = {CRBR},
  volume = {1},
  year = {1988},
  pages = {1-18}
}
\end{vb}

\citetest{7}{1-18}{14}{8}{petersen:1988}

\begin{refimp}
  7. David Petersen, “Hebrew Bible Textbooks: A Review Article,” \emph{CRBR} 1
  (1988): 1–18.

  14. Petersen, “Hebrew Bible Textbooks,” 8.

  \hangindent\bibindent Petersen, David. “Hebrew Bible Textbooks: A Review
  Article.” \emph{CRBR} 1 (1988): 1–18.
\end{refimp}

\subsubsection{An Unpublished Dissertation or Thesis}

\begin{vb}
@thesis{klosinski:1988,
  author = klosinski,
  title = {Meals in Mark},
  type = {phdthesis},
  institution = claremont,
  year = {1988}
}
\end{vb}  

\citetest{21}{22-44}{26}{23}{klosinski:1988}

\begin{refimp}
  21. Lee E. Klosinski, “Meals in Mark” (PhD diss., The Claremont Graduate
  School, 1988), 22–44.

  26. Klosinski, “Meals in Mark,” 23.

  \hangindent\bibindent Klosinski, Lee. E. “Meals in Mark.” PhD diss., The
  Claremont Graduate School, 1988.
\end{refimp}

\subsubsection{An Article in an Encyclopaedia or a Dictionary}

\begin{vb}
@inreference{stendahl:1962,
  author = stendahl,
  title = {Biblical Theology, Contemporary},
  shorttitle = {Biblical Theology},
  maintitle = IDB,
  shortmaintitle = {IDB},
  volume = {1},
  pages = {418-432}
}
\end{vb}  

\citetest{33}{418-432}{36}{419}{stendahl:1962}

\begin{refimp}
  33. Krister Stendahl, “Biblical Theology, Contemporary,” \emph{IDB}
  1:418–32.
  
  36. Stendahl, “Biblical Theology,“ 1:419.
  
  Stendahl, Krister. “Biblical Theology, Contemporary.” \emph{IDB} 1:418–32.
\end{refimp}

\subsubsection{An Article in a Lexicon or Theological Dictionary}

For the discussion of a word or a family of words, give the entire title and
page range of the article:

\begin{vb}
@mvlexicon{NIDNTT,
  editor = brown,
  title = NIDNTT,
  shorttitle = {NIDNTT},
  volumes = {4},
  location = grandrapids,
  publisher = zondervan,
  date = {1975/1985}
}

@inlexicon{dahn+liefeld:see+vision+eye,
  author = dahn # " and " # liefeld,
  title = {See, Vision, Eye},
  shortmaintitle = {NIDNTT},
  volume = {3},
  pages = {511-521},
  crossref = {NIDNTT},
  lexiconref = {NIDNTT}
}
\end{vb}

\citetestlex{3}{dahn+liefeld:see+vision+eye}

\begin{refimp}
  3. Karl Dahn and Walter L. Liefeld, “See, Vision, Eye,“ \emph{NIDNTT} 3:511–21.
\end{refimp}

\begin{vb}
@mvlexicon{TDNT,
  editor = kittel # " and " # friedrich,
  title = TDNT,
  title = TDNT,
  shorttitle = {TDNT},
  translator = bromily,
  volumes = {10},
  location = grandrapids,
  publisher = eerdmans,
  date = {1964/1976}
}

@inlexicon{beyer:diakoneo+diakonia+ktl,
  author = beyer,
  title = {\gr{διακονέω, διακονία, κτλ}},
  volume = {2},
  pages = {81-93},
  crossref = {TDNT},
  lexiconref = {TDNT}
}
\end{vb}
  
\citetestlex{6}{beyer:diakoneo+diakonia+ktl}

\begin{refimp}
  6. Hermann W. Beyer, “\gr{διακονέω, διακονία, κτλ},” \emph{TDNT} 2:81–93.
\end{refimp}

\begin{vb}
@mvlexicon{TLNT,
  author = spicq,
  title = TLNT,
  title = TLNT,
  shorttitle = {TLNT},
  editor = ernest,
  translator = ernest,
  volumes = {3},
  location = peabody,
  publisher = hendrickson,
  year = {1994}
}

@inlexicon{spicq:atakteo+ataktos+ataktos,
  author = spicq,
  title = {\gr{ἀτακτέω, ἄτακτος, ἀτάκτως}},
  volume = {1},
  pages = {223-224},
  crossref = {TLNT},
  lexiconref = {TLNT}
}
\end{vb}

\citetestlex{7}{spicq:atakteo+ataktos+ataktos}

\begin{refimp}
  7. Ceslas Spicq, “\gr{ἀτακτέω, ἄτακτος, ἀτάκτως},“ \emph{TLNT} 1:223–24.
\end{refimp}

\begin{vb}
@mvlexicon{TLNT,
  author = spicq,
  title = TLNT,
  title = TLNT,
  shorttitle = {TLNT},
  editor = ernest,
  translator = ernest,
  volumes = {3},
  location = peabody,
  publisher = hendrickson,
  year = {1994}
}

@inlexicon{spicq:amoibe,
  author = spicq,
  title = {\gr{ἀμοιβή}},
  volume = {1},
  pages = {95-96},
  crossref = {TLNT},
  lexiconref = {TLNT}
}
\end{vb}

\citetestlex{143}{spicq:amoibe}

\begin{refimp}
  143. Ceslas Spicq, “\gr{ἀμοιβή},” \emph{TLNT} 1:95–96.
\end{refimp}

\bigskip

For the discussion of a specific word in an article covering a larger group of
words, name just the word discussed and those pages on which it is discussed:

\begin{vb}
@mvlexicon{TDNT,
  editor = kittel # " and " # friedrich,
  title = TDNT,
  title = TDNT,
  shorttitle = {TDNT},
  translator = bromily,
  volumes = {10},
  location = grandrapids,
  publisher = eerdmans,
  date = {1964/1976}
}

@inlexicon{beyer:diakoneo,
  author = beyer,
  title = {\gr{διακονέω}},
  volume = {2},
  pages = {81-87},
  crossref = {TDNT},
  lexiconref = {TDNT}
}
\end{vb}

\citetestlex{23}{beyer:diakoneo}

\begin{refimp}
  23. Hermann W. Beyer, “\gr{διακονέω},” \emph{TDNT} 2:81–87.
\end{refimp}

\begin{vb}
@mvlexicon{NIDNTT,
  editor = brown,
  title = NIDNTT,
  shorttitle = {NIDNTT},
  volumes = {4},
  location = grandrapids,
  publisher = zondervan,
  date = {1975/1985}
}

@inlexicon{dahn:horao,
  author = dahn,
  title = {\gr{ὁράω}},
  volume = {3},
  pages = {511-518},
  crossref = {NIDNTT},
  lexiconref = {NIDNTT}
}
\end{vb}

\citetestlex{26}{dahn:horao}

\begin{refimp}
  26. Karl Dahn, “\gr{ὁράω},” \emph{NIDNTT} 3:511–18.
\end{refimp}

\bigskip

Subsequent entries need to include only the dictionary volume and page
numbers.

\citetestlexns{25}{83}{beyer:diakoneo+diakonia+ktl}

\begin{refimp}
  25. Beyer, \emph{TDNT} 2:83.
\end{refimp}

\citetestlexns{29}{511}{dahn:horao}

\begin{refimp}
  29. Dahn, \emph{NIDNTT} 3:511.
\end{refimp}

\citetestlexns{147}{95}{spicq:amoibe}

\begin{refimp}
  147. Spicq, \emph{TLNT} 1:95.
\end{refimp}

\bigskip

In the bibliography, cite only the theological dictionary.

\citetestbib{NIDNTT}

\begin{refimp}
  \hangindent\bibindent Brown, Colin, ed. \emph{New International Dictionary
  of New Testament Theology.} 4~vols. Grand Rapids: Zondervan, 1975–1985.
\end{refimp}

\citetestbib{TDNT}

\begin{refimp}
  \hangindent\bibindent Kittel, Gerhard, and Gerhard Friedrich, eds.
  \emph{Theological Dictionary of the New} Testament. Translated by Geoffrey
  W. Bromiley. 10~vols. Grand Rapids: Eerdmans, 1964–1976.
\end{refimp}

\citetestbib{TLNT}

\begin{refimp}
  \hangindent\bibindent Spicq, Ceslas. \emph{Theological Lexicon of the New
  Testament.} Translated and edited by James D. Ernest. 3~vols. Peabody, MA:
  Hendrickson, 1994.
\end{refimp}

\subsubsection{A Paper Presented at a Professional Society}

\begin{vb}
@inproceedings{niditch:1994,
  author = niditch,
  title = {Oral Culture and Written Documents},
  shorttitle = {Oral Culture},
  type = {presentedpaper},
  eventtitle = {the Annual Meeting of the New England Region
                of the SBL},
  location = worcester,
  day = {25},
  month = {3},
  year = {1994}
}
\end{vb}

\citetest{31}{13-17}{35}{14}{niditch:1994}

\begin{refimp}
  31. Susan Niditch, “Oral Culture and Written Documents” (paper presented at
  the Annual Meeting of the New England Region of the SBL, Worcester, MA, 25
  March 1994), 13–17.

  35. Niditch, “Oral Culture,” 14.

  \hangindent\bibindent Niditch, Susan. “Oral Culture and Written Documents.”
  Paper presented at the Annual Meeting of the New England Region of the SBL
  Worcester, MA, 25 March 1994.
\end{refimp}

\subsubsection{An Article in a Magazine}

\begin{vb}
@article{saldarini:1998,
  author = saldarini,
  title = {Babatha's Story},
  journaltitle = BAR,
  shortjournal = {BAR},
  volume = {24},
  number = {2},
  year = {1998},
  pages = {23-33, 36-37, 72-74}
}
\end{vb}  

\citetest{8}{23-33, 36-37, 72-74}{27}{28}{saldarini:1998}

\begin{refimp}
  8.Anthony J. Saldarini, “Babatha’s Story,“ \emph{BAR} 24.2 (1998): 28–33,
  36–37, 72–74.

  27. Saldarini, “Babatha’s Story,” 28.

 \hangindent\bibindent Saldarini, Anthony J. “Babatha’s Story.” \emph{BAR}
 24.2 (1998): 28–33, 36–37, 72–74.
\end{refimp}

\subsubsection{An Electronic Journal Article}

\begin{vb}
@article{springer:2014,
  author = springer,
  title = {Of Roosers and \emph{Repetitio}: Ambrose’s
           \emph{Aeterne rerum conditor}},
  journal = VC,
  shortjournal = {VC},
  volume = {68},
  year = {2014},
  pages = {155-177},
  doi = {10.1163/15700720-12341158}
}
\end{vb}

\citetest{43}{155-177}{45}{158}{springer:2014}

\begin{refimp}
  43. Carl P. E. Springer, “Of Roosters and \emph{Repetitio}: Ambrose’s
  \emph{Aeterne rerum conditor},“ \emph{VC} 68 (2014): 155–77,
  \url{doi:10.1163/15700720-12341158}.

  45. Springer, “Of Roosters and \emph{Repetitio},” 158.

  \hangindent\bibindent Springer, Carl P. E. “Of Roosters and
  \emph{Repetitio}: Ambrose’s \emph{Aeterne rerum conditor}.” \emph{VC}
  68(2014):155–77. \url{doi:10.1163/15700720-12341158}.
\end{refimp}

\begin{vb}
@article{truehart:1996,
  author = truehart,
  title = {Welcome to the Next Church},
  shorttitle = {Next Church},
  url = {http://www.theatlantic.com/atlantic/issues/
         96aug/nxtchrch/nxtchrch.htm},
  journaltitle = atlanticmonthly,
  volume = {278},
  month = {8},
  year = {1996},
  pages = {37-58}
}
\end{vb}

\citetest{8}{37-58}{12}{37}{truehart:1996}

\begin{refimp}
  8. Charles Truehart, “Welcome to the Next Church,” \emph{Atlantic Monthly}
  278 (August 1996): 37–58,
  \url{http://www.theatlantic.com/past/docs/issues/96aug/nxtchrch/nxtchrch.htm}.

  12. Truehart, “Next Church,” 37.
 
  \hangindent\bibindent Truehart, Charles. “Welcome to the Next Church.”
  \emph{Atlantic Monthly} 278 (August
  1996): 37–58.
  \url{http://www.theatlantic.com/past/docs/issues/96aug/nxtchrch/nxtchrch.htm}.
\end{refimp}

\begin{vb}
@article{kirk:2007,
  author = kirk,
  title = {Karl Polanyi, Marshall Sahlins, and the Study of
           Ancient Social Relations},
  shorttitle = {Karl Polanyi},
  journal = JBL,
  shortjournal = {JBL},
  volume = {126},
  year = {2007},
  pages = {182-191},
  doi = {10.2307/27638428},
  url = {http://www.jstor.org/stable/27638428}
}
\end{vb}

\citetest{31}{182-191}{35}{186}{kirk:2007}

\begin{refimp}
  31. Alan Kirk, “Karl Polanyi, Marshall Sahlins, and the Study of Ancient Social
  Relations,” \emph{JBL} 126 (2007): 182–91, \url{doi:10.2307/27638428},
  \url{http://www.jstor.org/stable/27638428}.

  35. Kirk, “Karl Polanyi,” 186.

  \hangindent\bibindent Alan Kirk. “Karl Polanyi, Marshall Sahlins, and the
  Study of Ancient Social Relations,” \emph{JBL} 126 (2007): 182–91.
  \url{doi:10.2307/27638428}. \url{http://www.jstor.org/stable/27638428}.
\end{refimp}

\subsection{Special Examples}

\subsubsection{Texts from the Ancient Near East}

\textbf{6.4.1.1 Citing \emph{COS}}

TODO: Not yet supported.

\begin{refimp}
  7. “The Great Hymn to the Aten,” trans. Miriam Lichtheim (\emph{COS}
  1.26:44-46).

  11. “Great Hymn to the Aten,” \emph{COS} 1.26:44–46.

  \hangindent\bibindent Hallo, William W., ed. \emph{Canonical Compositions
  from the Biblical World.} Vol.~1 of \emph{The Context of Scripture.} Leiden:
  Brill, 1997.
\end{refimp}

\textbf{6.4.1.2 Citing Other Texts}

\begin{refimp}
  16. “Suppiluliumas and the Egyptian Queen,” trans. Albrecht Goetze
  (\emph{ANET,} 319).

  \hangindent\bibindent Pritchard, James B., ed. \emph{Ancient Near Eastern
  Texts Relating to the Old Testament.} 3rd ed. Princeton: Princeton
  University Press, 1969.
\end{refimp}

\begin{refimp}
  5. “Erra and Ishum” (Stephanie Dalley, \emph{Myths from Mesopotamia}
  [Oxford: Oxford University Press, 1991], 282–315).
  
  \hangindent\bibindent Dalley, Stephanie. \emph{Myths from Mesopotamia.}
  Oxford: Oxford University Press, 1991.
\end{refimp}

\begin{refimp}
  5. “Erra and Ishum” (Benjamin Foster, \emph{Before the Muses: An Anthology
  of Akkadian Literature} [Bethesda, MD: CDL, 1993], 1:771–805).
  
  \hangindent\bibindent Foster, Benjamin. \emph{Before the Muses: An Anthology of
  Akkadian Literature.} Vol.~1 Bethesda, MD: COL, 1993.
\end{refimp}

\begin{refimp}
  34. “The Doomed Prince” (Miriam Lichtheim, \emph{Ancient Egyptian
  Literature} [Berkeley: University of California Press, 1976], 2:200–203).
  
  36. “The Doomed Prince” (\emph{AEL} 2:200–203).
  
  \hangindent \bibindent Lichtheim, Miriam. Ancient Egyptian Literature.
  Vol~2. Berkeley: University of California Press, 1976.
\end{refimp}

\begin{refimp}
  12. “The Disappearance of the Sun God,” §3 (A I 11–17) (Harry A. Hoffner
  Jr., \emph{Hittite Myths} [ed. Gary M. Beckman; WAW 2; Atlanta: Scholars
  Press, 1990], 26).
  
  \hangindent\bibindent Hoffner, Harry A., Jr. \emph{Hittite Myths.} Edited by
  Gary M. Beckman. WAW 2. Atlanta: Scholars Press, 1990.
\end{refimp}

\begin{refimp}
  32. Ashur Inscription, obv.\ lines 10–17 (Albert Kirk Grayson,
  \emph{Assyrian Rulers of the Early First Millennium BC [1114–859 BC],} RIMA
  2 [Toronto: University of Toronto Press, 1991], 143–44).

  \hangindent\bibindent Grayson, Albert Kirk. \emph{Assyrian Rulers of the
  Early First Millennium BC (1114-859 BC).} RIMA 2. Toronto: University of
  Toronto Press, 1991.
\end{refimp}

\begin{refimp}
  33. Esarhaddon Chronicle, lines 3–4 (Albert Kirk Grayson, \emph{Assyrian and
  Babylonian Chronicles,} TCS [Locust Valley; NY: Augustin, 1975], 125).

  33. Esarhaddon Chronicle, lines 3-4 (ABC, 125).

  \hangindent\bibindent Grayson, Albert Kirk. \emph{Assyrian and Babylonian
  Chronicles.} TCS. Locust Valley; NY: Augustin, 1975.
\end{refimp}

\begin{refimp}
  45. ARM 1.3.

  \hangindent\bibindent Dossin, Georges. \emph{Lettres.} ARM 1.1946. Repr.,
  Paris: Geuthner, 1967.
\end{refimp}

\begin{refimp}
  45. ARMT 1.3.

  \hangindent\bibindent Georges Dossin, \emph{Correspondance de Šamši-Addu et de
  ses fils.} ARMT 1. Paris: Imprimerei nationale, 1950.
\end{refimp}

\subsubsection{Loeb Classical Library (Greek and Latin)}

TODO: Not yet supported.

\begin{refimp}
  (Josephus, \emph{Ant.} 2.233–235)

  1. Josephus, \emph{Ant.}\ 2.233–235.
\end{refimp}

\begin{refimp}
  4. Tacitus, \emph{Ann.}\ 15.18–19
\end{refimp}

\begin{refimp}
  (Josephus, \emph{Ant.}\ 2.233–235 [Thackeray, LCL])

  5. Josephus, \emph{Ant.}\ 2.233–235 (Thackeray, LCL).
\end{refimp}

\begin{refimp}
  6. Tacitus, \emph{Ann.}\ 15.18–19 (Jackson, LCL).
\end{refimp}

\begin{refimp}
  \hangindent\bibindent \emph{Josephus.} Translated by Henry St. J. Thackeray
  et al. 10 vols. LCL Cambridge: Harvard University Press, 1926–1965.

  \hangindent\bibindent Tacitus. \emph{The Histories and The Annals.}
  Translated by Clifford H. Moore and John Jackson. 4 vols. LCL Cambridge:
  Harvard University Press, 1937.
\end{refimp}

\begin{refimp}
  14. Flavius Josephus, \emph{The Jewish Antiquities, Books 1–19,} trans.\
  Henry St.]. Thackeray et al., LCL (Cambridge: Harvard University Press,
  1930–1965).

  \hangindent\bibindent \emph{Josephus.} Translated by Henry St. J. Thackeray
  et al. 10 vols. LCL Cambridge: Harvard University Press, 1926–1965.
\end{refimp}

\subsubsection{Papyri, Ostraca, and Epigraphica}

TODO: Not yet supported.

\textbf{6.4.3.1 Papyri and Ostraca in General}

\begin{refimp}
  (P.Cair.Zen.\ 59003)

  22. P.Cair.Zen.\ 59003.

  22. P.Cair.Zen.\ 59003 (Arthur S. Hunt and Campbell C. Edgar, \emph{Select
  Papyri,} LCL [Cambridge: Harvard University Press, 1932], 1:96).

  22. P.Cair.Zen. 59003 (Hunt and Edgar §31).
\end{refimp}

\textbf{6.4.3.2 Epigraphica}

\textbf{6.4.3.3 Greek Magical Papyri}

\begin{refimp}
  (PGM III. 1–164)

  22. PGM III. 1–164.

  22. PGM IIL 1-164 (Betz).

  \hangindent\bibindent Betz, Hans Dieter. \emph{The Greek Magical Papyri in
  Translation, Including the Demotic Spells.} 2nd ed. Chicago: University of
  Chicago Press, 1996.
\end{refimp}

\subsubsection{Ancient Epistles and Homilies}

TODO: Not yet supported.

\begin{refimp}
  (Heraclitus, \emph{Epistle 1,} 10)

  34. Heraclitus, \emph{Epistle 1,} 10.

  36. Heraclitus, \emph{Epistle 1,} 10 (Worley).

  \hangindent\bibindent Heraclitus. \emph{Epistle 1.} Translated by David
  Worley. Page 187 in \emph{The Cynic Epistles: A Study Edition.} Edited by
  Abraham J. Malherbe. SBS 12. Atlanta: Scholars Press, 1977.

  \hangindent\bibindent Malherbe, Abraham J., ed. \emph{The Cynic Epistles: A
  Study Edition.} SBS 12. Atlanta: Scholars Press, 1977.
\end{refimp}

\subsubsection{\emph{ANF} and \emph{NPNF}, First and Second Series}

TODO: Not yet supported.

\begin{refimp}
  14. \emph{The Clementine Homilies} 1.3 (ANF 8:223).

  \hangindent\bibindent \emph{The Ante-Nicene Fathers.} Edited by Alexander
  Roberts and James Donaldson. 1885–1887. 10 vols. Repr., Peabody, MA:
  Hendrickson, 1994.
\end{refimp}

\begin{refimp}
  44. Augustine, \emph{Letters of St. Augustin} 28.3.5
  (NPNF\textsuperscript{1} 1:252).
  
  \hangindent\bibindent Augustine. \emph{The Letters of St. Augustin.} In
  vol.~1 of \emph{The Nicene and Post-Nicene Fathers,} Series 1. Edited by
  Philip Schaff. 1886–1889. 14 vols. Repr., Peabody, MA: Hendrickson, 1994.
\end{refimp}

\subsubsection{J.-P. Migne's Patrologia Latina and Patrologia Graeca}

TODO: Not yet supported.

\begin{refimp}
  6. Gregory of Nazianzus, \emph{Orationes theologicae} 4 (PG 36:12c).

  \hangindent\bibindent Patrologia Latina. Edited by J.-P. Migne. 217 vols.
  Paris, 1844–1864.

  \hangindent\bibindent Patrologia Graeca. Edited by J.-P. Migne. 162 vols.
  Paris, 1857–1886.
\end{refimp}

\subsubsection{Strack-Billerbeck,\newline\emph{Kommentar zum Neuen Testament}}

TODO: Not yet supported.

\begin{refimp}
  3. See the discussion of \gr{ἐκρατοῦντο} in Str-B 2:271.

  \hangindent\bibindent Strack, Hermann L., and Paul Billerbeck.
  emph{Kommentar zum Neuen Testament aus Talmud und Midrasch.} 6 vols. Munich:
  Beck, 1922–1961.
\end{refimp}

\subsubsection{\emph{Aufsteig und Niedergang der römischen\newline Welt (ANRW)}}

TODO: Note yet supported.

\begin{refimp}
  76. Graham Anderson, “The \emph{pepaideumenos} in Action: Sophists and Their
  Outlook in the Early Empire,” \emph{ANRW} 33.1:80–208.

  79. Anderson, “\emph{Pepaideumenos,}” \emph{ANRW} 33.1:86.
\end{refimp}

\begin{refimp}
  \hangindent\bibindent Temporini, Hildegard, and Wolfgang Haase, eds.
  \emph{Aufstieg und Niedergang der romischen Welt: Geschichte und Kultur Roms
  im Spiegel der neueren Forschung.} Part 2, \emph{Principat.} Berlin: de
  Gruyter, 1972–.
\end{refimp}

\subsubsection{Bible Commentaries}

\begin{vb}
@book{hooker:1991,
  author = hooker,
  title = {The Gospel according to Saint Mark},
  series = BNTC,
  shortseries = {BNTC},
  number = {2},
  location = peabody,
  publisher = hendrickson,
  year = {1991}
}
\end{vb}  

\citetestns{6}{223}{hooker:1991}

\begin{refimp}
  8. Morna Hooker, \emph{The Gospel according to Saint Mark,} BNTC 2 (Peabody,
  MA: Hendrickson, 1991), 223.

  \hangindent\bibindent Hooker, Morna. \emph{The Gospel according to Saint
  Mark.} BNTC 2. Peabody, MA: Hendrickson, 1991.
\end{refimp}

\subsubsection{A Single Volume of a Multivolume Commentary in a Series}

The style for citing a single volume of a multivolume commentary in a series
is the same as for a titled volume in a multivolume edited work (§6.2.21). The
style for citing the entire work follows that for a multivolume work
(§6.2.20).

\subsubsection{SBL Seminar Papers}

\begin{vb}
@seminarpaper{crenshaw:2001,
  author = crenshaw,
  title = {Theodicy in the Book of the Twelve},
  booktitle = {Society of Biblical Literature 2001 Seminar Papers},
  series = SBLSP,
  shortseries = {SBLSP},
  number = {40},
  location = atlanta,
  publisher = sbl,
  year = {2001},
  pages = {1-18}
}
\end{vb}

\citetestns{33}{1-18}{crenshaw:2001}

\begin{refimp}
  33. James L. Crenshaw, “Theodicy in the Book of the Twelve,” \emph{Society
  of Biblical Literature 2001 Seminar Papers,} SBLSPS\footnote{Should be
  SBLSP?} 40 (Atlanta: Society of Biblical Literature, 2001), 1–18.
  
  \hangindent\bibindent Crenshaw, James L. “Theodicy in the Book of the
  Twelve.” Pages 1–18 in \emph{Society of Biblical Literature 2001 Seminar
  Papers.} SBLSPS\textsuperscript{\emph{a}} 40. Atlanta: Society of Biblical
  Literature, 2001.
\end{refimp}

\subsubsection{A CD-ROM Reference (with a Corresponding Print Edition)}

Books on CD-ROM should be cited according to the print edition. It is not
necessary to indicate the medium in the citation.

\subsubsection{Text Editions Published Online with No Print Counterpart}

TODO: Not yet supported.

\begin{refimp}
  2. Gernot Wilhelm, ed., “Der Vertrag Šuppiluliumas I. von Ḫatti mit
  Šattiwazza von Mitrani (CTH 51.I),” released 24 February 2013,
  doi:hethiter.net/: CTH 51.I (INTR 2013-02-24).

  4. Wilhelm, “Der Vertrag Šuppiluliumas I.”

  \hangindent\bibindent Gernot Wilhelm, ed. “Der Vertrag Šuppiluliumas I. von
  Ḫatti mit Šattiwazza von Mitrani (CTH 51.I).” doi:hethiter.net/: CTH 51.I
  (INTR 2013-02-24).
\end{refimp}

\subsubsection{Online Database}

TODO: Not yet supported.

\begin{refimp}
  37. Cobb Institute of Archaeology. “The Figurines of Maresha, the Persian
  Era,” DigMaster,
  \url{http://www.cobb.msstate.edu/dignew/Maresha/index.html}.

  \hangindent\bibindent Cobb Institute of Archaeology. “The Figurines of
  Maresha, the Persian Era.” DigMaster.
  \url{http://www.cobb.msstate.edu/dignew/Maresha/index.html}.
\end{refimp}

\subsubsection{Websites and Blogs}

\begin{refimp}
  \raggedright 

  10. “The One Hundred Most Important Cuneiform Objects,” cdli:wiki,
  \url{http://cdli.ox.ac.uk/wiki/doku.php?id=the_one_hundred_most_important_cuneiform_objects}.

  \hangindent\bibindent “The One Hundred Most Important Cuneiform Objects.”
  cdli:wiki.
  \url{http://cdli.ox.ac.uk/wiki/doku.php?id=the_one_hundred_most_important_cuneiform_objects}.
\end{refimp}

\printbibliography

\end{document}
